\documentclass{article}
\usepackage{amssymb}
\usepackage{amsmath}
\usepackage{algorithm}
\usepackage{algpseudocode}
\usepackage{pgfplots}
\usepackage{multicol}
\usepackage[hmarginratio=1:1,top=32mm,columnsep=20pt]{geometry}
\usepackage{fullpage}
\usepackage{pdflscape}
\usepackage[toc,page]{appendix}
\usepackage[shortlabels]{enumitem}
\usepackage[font=itshape]{quoting}
\usepackage{tabularx} % for 'tabularx' environment and 'X' column type
\usepackage{makecell}
\usepackage[hidelinks]{hyperref}
\usepackage{setspace}

\begin{document}
\parindent=0in
\parskip=12pt

\title{
  Toward a New Great Compromise - DRAFT \\
  \large{
    A Constitutional Plan to Promote Liberty, Renew Democracy, and Restore Federalism
  }
}

\author{Max Sklar, Local Maximum Labs, July, 2023}
\date{}

\maketitle

The Constitution of the United States has not had a structural amendment for many decades. A convention of the states that could make this happen has never occurred. However, the contradictions in any political structure tend to become more pronounced over time, and the US is no exception. This means that someday, someone or some group will be in a position to make major changes. If well-meaning Americans fail to discuss and debate how our government is structured on the most basic level, these decisions will one day be left in the hands  of nefarious actors.

Our approach should neither deify the founding fathers nor dismiss them as irrelevant relics. We should instead seek to understand why they made the decisions as they did, and how those decisions have echoed throughout our history. It is through this understanding that we can evaluate future changes.

\section{The Emerging Crisis}

Americans demand a lot from their form of government. We say that it should be a federal republic with checks and balances to maintain the rule of law. It should be democratic both in the derivation of its power through its citizens and in its approach to transparency and open debate. This includes the ability for multiple parties to compete and respect for civil liberties including the all-important bill of rights.

Many Americans watch in horror as they find their liberties under attack and liberal democracy slipping away. Due to extreme polarization, they cannot even agree on where the problem lies. This fracture partially sets centralized progressive urban populations against widely distributed rural ones. 

Neither side is going away, and neither has enough support to permanently impose their will on the other. In the spirit of the “Great Compromise” of 1787 that founded this country, we seek to create a plan that has properties attractive to both the left and the right, and that could garner broad support after public debate. This work provides a rough attempt at this. Every conversation for great improvement begins as partially developed. I seek your feedback for improvements.

\section{A New Great Compromise}

The rural faction would be well served by a repeal of the 17th amendment\cite{Tucker}\cite{Virginia} which required the direct election of senators. With its repeal, the senate would be selected by the state legislatures. This would presumably lead to a less "top heavy" federal structure with states having more control over local issues. However, the 17th amendment will undoubtedly find support among the progressive faction which has benefited from urban votes and prefers popular votes with large constituencies\cite{Smith}.

Instead of repealing this amendment, this proposal seeks a comprehensive change to the structure of government that is both principled and satisfactory to the various factions. It includes better representation for the state governments than a repeal of the 17th amendment could ever hope to achieve, along with apportionment reforms long sought after by the progressive faction. Furthermore, an executive council will be created to solve long-standing shortcomings of the current structure, including taking back a degree of control and accountability from the permanent bureaucracy.

Institutions degrade over time as they attract special interests and ideologues who would like to use them for their own purposes. To combat this, institutions should have a clear and believable purpose. This provides a schelling point, or a default manner in decision-making, around which its members can agree and refer to\cite{Komoroske}. 

The delegates at the 1787 constitutional convention had competing ideas for the Senate and differed on whether it was going to be the “voice of the states” or the “more professional and permanent upper chamber”. They ended up combining both. This double duty backfired, preventing the Senate from optimally serving either purpose. This confusion has echoed throughout history, eventually leading to the 17th amendment.

By separating out these concerns into a Senate and an Execute Council, we can have a government with democracy, federalism, and continuity. As we shall see, there is quite a bit of support for this structure historically. It can be found in successful world and state governments, the private sector, and even from the founders themselves.

The changes are summarized as follows:
\begin{enumerate}
  \item The \textbf{House of Representatives} will include representation for DC and the territories, and a stronger judicial oversight to evaluate election law and district maps.
  \item Members of the \textbf{Senate} are chosen by their respective states and are in office at-will, thereby establishing the accountability they currently lack.
  \item An \textbf{Executive Council} will be established to replace those aspects of the senate that require continuity and long term lengths. Selected by a senate-like electoral college, this council will provide professional governance for the executive branch. The first national Condorcet election will be introduced in order to encourage independent candidates.
  \item The electoral college for the \textbf{President} will be apportioned by population like the house of representatives.
 \item Each institution is given a clear purpose and mission.
\end{enumerate}

The rest of this document will explain the reasons for these decisions though the drafting of a constitutional amendment.

\section{House of Representatives}

\begin{quoting}
\textbf{Section 1}

The purpose of the house of representatives is to represent in its members the broad spectrum of interests of the people of the United States in the legislative branch.
\end{quoting}

Elected representatives have a clear conflict of interest when it comes to election reform because any change will come with advantages and disadvantages for their own success. The supreme court has therefore taken a strong role in election law, but they continue to grapple with defining that role.

The recent decision in Allen vs. Milligan\cite{Allen_Milligan} is an example of where the court can intervene in unfair electoral district maps. It is certainly not the first. However, this intervention has been fairly limited\cite{Brewer} because while the courts have legal basis to protect against racial gerrymandering, they cannot do much about unfair partisan advantage. We seek to expand the role of the courts here.

\begin{quoting}
The manner in which states elect their representatives shall be designed to produce a delegation that is broadly representative of the population of that state and its various communities within the constraint of the delegation size. This manner of election falls first to the states, and then to federal law which may make or alter such regulations.

Any group of electors shall have standing to bring suit in the federal judiciary with claims that these election regulations are misaligned with their constitutional purpose. In considering such cases, the judiciary shall take into account the totality of
\begin{itemize}
\item the historical precedent
\item the changing facts, particularly concerning the electorate, which may degrade a once-valid electoral system
\item our evolving understanding of the empirical properties of electoral law as discovered through observation of elections in practice.
\item our evolving understanding of the theoretical properties of electoral law as discovered through logical deduction
\end{itemize}
\end{quoting}

This would allow judges to take a more active stand on election law. Over time, the courts will have a freer hand in correcting problems with redistricting and partisan gerrymandering. They will use past decisions as a starting point, but will not be wholly constrained by them.

Electoral systems are a hard problem, and no magic solution exists to achieve mathematical perfection. Some might say an abolition of districts altogether in favor of a proportional system would eliminate these problems. They may be right, but there are some potential pitfalls to such a system as well.

We can instead set ourselves up to evolve with a north star statement like: “broadly representative of the population of that state and its various communities”. Courts will grapple with the idea of “various communities” as each group presents their case for representation, and will come to a set of principles for evaluating these claims over time.

With that established, it makes sense to ensure that all citizens get a voice in the house of representatives, including those who do not reside in any state.

\begin{quoting}
In addition to apportioning seats in the house of representatives to each state based on population, seats shall be apportioned for the District of Columbia and United States territories. The seats allocated to each entity shall be based on its population just the same as if it were a state. Territories with populations less than half of the number of house seats per capita shall instead receive a non-voting member.
\end{quoting}

This provides representation to the population in Puerto Rico and DC. This would mean approximately 1 seat for DC and 3 seats for Puerto Rico, with the smaller territories receiving non-voting members.

Congress should consider increasing the size of the House in order to better represent the large population of the United States. Several plans have been introduced in order to do this, from a modest increase to a gargantuan one\cite{Allen}. A plan provided by thiry-thousand.org even calls for an increase in the house in order to satisfy the low population-to-representative ratio envisioned at constitutional convention\cite{30000}.

Madison urges caution in Federalist 10 when he writes, ”the representatives must be raised to a certain number, in order to guard against the cabals of a few; and that, however large it may be, they must be limited to a certain number, in order to guard against the confusion of a multitude.”\cite{Federalist10} With that in mind, appendix~\ref{appendix:house} provides a sketch of how the house could be restructured with more voices without increasing its size.

\section{Senate}
\label{section:Senate}

The constitutional convention was split between the Senate existing to “represent the states” and to be a “long-term, more professional, upper house”. Hamilton and Madison saw it as the latter, with Hamilton originally proposing life-long terms, and Madison originally proposing that they be appointed by the house.

Even the delegates who proposed senators be elected by the state legislatures were not motivated entirely by concern for the states. John Dickinson ”wished the Senate to consist of the most distinguished characters, distinguished for their rank in life and their weight of property, and bearing as strong a likeness to the British House of Lords as possible; and he thought such characters more likely to be selected by the State Legislatures, than in any other mode.”\cite{Madison}

Dickinson prevailed at the convention. Madison promoted the state legislature appointment of senators in Federalist 62 when he wrote, “It is recommended by the double advantage of favoring a select appointment, and of giving to the State governments such an agency in the formation of the federal government as must secure the authority of the former, and may form a convenient link between the two systems.”\cite{Federalist62}

Both Madison and Dickinson cared about securing a degree of agency for the states, but they often presented it as an added bonus rather than sole reason. By combining several competing conceptions of the senate, the framers never gave a consistent message. It is no wonder that the 17th amendment was ratified without much debate around this!

The 17th amendment was passed by in 1912 with the hopes that direct elections of senators would lead to more democratic accountability\cite{Eisinger}. Has it delivered on its promise? A report by the Brookings Institute on the 100th anniversary of the 17th amendment found that none of this accountability could be verified by political data\cite{Schiller}. Senators elected directly seem to be no more responsive to voters than senators elected indirectly.

And why should they be? They have a six year term. Their constituents' voting power is diluted among millions of people, and they can more easily win elections by courting doners than by listening to individual voters\footnote{As the character Pappy O'Daniel said in Oh Brother, Where art Thou, ”We ain't one-at-a-timin' here. We're MASS communicatin'!”}. The progressive authors of the 17th amendment mistakenly believed that it was the indirect election of the senators that made them unaccountable without considering the term length.

The convention delegates considered term length as an important feature of the institutions they created. They understood that there were trade-offs; that longer terms led to independence and shorter terms led to accountability. Roger Sherman, Connecticut delegate and author of the “Great Compromise“ warned against long term lengths when he said, “The more permanency it has the worse if it be a bad government.”\cite{Madison}

To ameliorate this confusion around the Senate's purpose, we separate it into 2 parts: the new Senate will represent the states, whose members are now subject to at-will employment.  Then we will create an Executive Council to make up for what this new Senate lacks.

\begin{quoting}
\textbf{Section 2}

The purpose of the Senate is to represent the interests of the individual states in the legislative branch.

Each state shall appoint one senator, and may appoint any number of alternate senators. Such alternates may engage in Senate business in the senator’s absence, according to a clear order of succession indicated by their constituent state.
\end{quoting}

This section starts out with a change to article 1 section 3 on the composition of the senate and renders the 17th amendment inoperative. Some senators today exist only to provide party-line votes without any talent as a legislator. Others are political relics who can coast from familiarity with voters and donors. This new Senate would contain fewer of these types, and its members would need to closely monitor the effects of federal legislation on their particular state government. As a result, federal laws would not encroach on the states as much as they currently do. Unfunded mandates would come under particular scrutiny.

The founders wanted multiple senators because otherwise there would be too many absences and states not represented. Nowadays, ease of communication and travel make absences less necessary during key votes. Still, senators will not always be available, so we will allow alternates. At the constitutional convention, Maryland delegate Martin Luther called for 1 vote per state, as it was under the Continental Congress and the Articles of Confederation\cite{Senate}.

Interestingly, this plan does not rely on the equal suffrage of states in the Senate. We keep the equal suffrage to remain consistent with article V of the constitution. However, some amount of reapportionment in order to satisfy the progressive faction would not be out of the question, so long as there is a viable plan for doing so constitutionally\cite{Amar}. A modest plan could allocate half the senate equally by state and the other half by population. That would dampen extreme population differences without giving a small number of states an enormous block. This can be supported in a principled way and is not just a split-the-baby compromise. Mathematical analysis of electoral systems, such as the one developed by Shapley and Shubik\cite{Shapley}, show that block voting is worth more than the sum of its parts\cite{Gross}.

\begin{quoting}
The method of selecting and removing senators shall be determined by the laws of the state from which they represent. A state may appoint, recall, or replace their senate delegation at any time in accordance with the laws of that state. The senate shall determine reasonable and orderly procedures for updating its roster and handling rotations as it receives notices of appointment, recall, and replacement from the states.

Absent relevant state laws, senators shall be selected by the state legislature. If the state legislature does not select a senator, the highest executive officer in the state can appoint a senator to serve in the interim. This method of appointment is overruled only by state laws passed after this amendment is ratified.
\end{quoting}

The authors of the 17th amendment were concerned about deadlocks that occurred where no senator was chosen. This solves it.

We next change article 1 section 6 of the constitution concerning the salaries of senators.

\begin{quoting}
Senators and alternate senators shall not receive any compensation from the treasury of the United States. They may receive compensation from the state they represent, as prescribed by the laws of their constituent state.
\end{quoting}

Madison wanted senators to have long terms in order to provide stability and continuity in government.\cite{Senate} Because of this, salaries needed to come from the federal government so that a state could not effectively recall a senator (or exert undue influence) by halting payments. Now that the senator serves at the pleasure of their state, that senator can get paid by the state.

\begin{quoting}
The senate shall retain the power to provide advice and consent on justices of the supreme court, to confirm ambassadors and other public ministers, to approve treaties, to try impeachments, and their full role in the legislative process.
\end{quoting}

\section{Executive Council}

Today we would be hard pressed to find a reason to support a “House of Lords“ as some founders wanted.

Instead, we can recognize the value in having elected positions where terms are long and removal is unlikely. It allows these officeholders to focus on the job at hand rather than reelection campaign, and to pursue long term goals that could replicate the decision-making processes of owners of the county's assets rather than renters. It also provides a sense of continuity and stability to allow for institutional knowledge to be retained.

The Senate was originally tasked with this role. Under this proposal, the Senate may have some members with long tenure, but they will be continually held accountable to their state. So, we create the executive council and place it in the executive branch to fulfill this role. The Executive Council will be modeled as a board of directors, an institution that has been tested in the private sector for many years in order to be proper stewards of shareholder value.

According to a study in the Wall Street Journal, small boards tend to be the most effective at driving high returns\cite{Lublin}. Here we create a board of 6 councilors and the president who act as a team. They may differ in their style and philosophical outlook, but they have the same constituency and mission. This is similar to how a board of directors has a fiduciary duty to act in the interests of the common shareholder.

This contrasts with the house and the senate whose members represent adversarial interests. The powers of the executive council must be debated thoroughly because although elected democratically, the council does not make a good replacement for democratic institutions such as the house. Instead, it should be used as an assist to ensure that the government's assets are well cared for over the long run.

\begin{quoting}
\textbf{Section 3}

The Executive Council’s purpose is to protect the long term interests and well-being of the United States.

The Executive Council shall consist of 6 councilors and the President of the United States, all with an equal vote on the council. The president may not vote until all of the other councilors have voted, and only if such a vote would affect the outcome of the resolution.
\end{quoting}

This is so that the president does not need to attend most of the Council meetings, and can be presented with resolutions after the fact.

\begin{quoting}
Each counselor shall serve for a term of 6 years, with staggered terms so that one term ends and another simultaneously begins once per year, with a more precise schedule set by law. Eligibility for councilor shall be the same as eligibility for senator.
\end{quoting}

In creating this council, we can take lessons from other councils and private sector boards\cite{Khanna}. The executive council of New Hampshire, for example, has been particularly effective as the “board of directors”\cite{Hahn-Burkett}. Although it is made up of Democrats and Republicans, the debates are less partisan\cite{Timmins}. More research is needed to determine a causal link, but New Hampshire has maintained high levels of economic freedom and fiscal discipline in relation to similar states\cite{Ruger}.

The government of Switzerland is also run by a federal council, with the Swiss being generally satisfied with their form of government and democratic institutions according to the Organisation for Economic Co-operation and Development\cite{Kaufman}.

There are also cautionary tales in council design. The defunct New York City Board of Estimate was made up of officials from other elected positions and its powers included permissions for permits and land use that encouraged trading political favors rather than expertise\cite{Purnick}. It was ultimately ruled unconstitutional for its malapportionment\cite{Board_of_Estimate}.

\subsection{The Proposals and Prophecies of George Mason}

George Mason was a founding father at the constitutional convention. He refused to endorse the constitution until it included a bill of rights, for which he was a strong advocate.

He wrote disapprovingly of a unitary executive when he said, ”If strong and extensive Powers are vested in the Executive, and that Executive consists only of one Person, the Government will of course degenerate... into a Monarchy: A Government so contrary to the Genius of the People, … These Sir, are some of my Motives for preferring an Executive consisting of three Persons rather than of one.”\cite{Mason}

The convention decided against Mason's original proposal in favor of a unitary executive. In doing so, they may have observed that most effective organizations rely on some kind of chief executive. As the convention came to a close, Mason implored the delegates to at least have the best of both worlds by Providing the president with a council.

He writes: “The President of the United States has no Constitutional Council, a thing unknown in any safe and regular government. He will therefore be unsupported by proper information and advice, and will generally be directed by minions and favorites; or he will become a tool to the Senate--or a Council of State will grow out of the principal officers of the great departments; the worst and most dangerous of all ingredients for such a Council in a free country; From this fatal defect has arisen the improper power of the Senate in the appointment of public officers, and the alarming dependence and connection between that branch of the legislature and the supreme Executive.“\cite{Mason_Objection}

Here, Mason points out the confusion that the framers had with respect to the senate by making it a legislative body with executive characteristics. His warning of a “Council of State“ seems particularly precinct today in light of the vast powers of the intelligence agencies and permanent administrative state.\cite{Cooper}

In Mason's final proposal to the convention, “...he suggested that a privy Council of six members to the president should be established; to be chosen for six years by the Senate, two out of the Eastern two out of the middle, and two out of the Southern quarters of the Union, \& to go out in rotation two every second year; the concurrence of the Senate to be required only in the appointment of Ambassadors, and in making treaties. which are more of a legislative nature.“\cite{Madison}

He was careful to note the differences between the Senate and the Council, offering advice on which should do what. He “considered the Senate as too unwieldy \& expensive for appointing officers, especially the smallest...“

Our proposal - like George Mason's - envisions a council that turns over gradually. However, we note that representation by region is already achieved through the house and the senate. Instead, the differentiating factor in council appointments will be time. Elections suffer from a recency bias, where their outcomes tend to be disproportionately affected by events closer to the election date. There's a reason why the phrase “October Surprise“ and not “June Surprise“ is used in relation to the presidential election. By staggering its terms, the council can counterbalance the recency bias and short-lived political fads\footnote{Every well engineered information system must account for anomalies and trends, something that I am familiar with through designing recommendation systems.\cite{Yang}} that come from the electoral process.

\subsection{Electing Councilors}

We now turn to a unique method of electing councilors, which is designed to give the council the properties it needs to serve its purpose well.

\begin{quoting}
The council election shall occur in several steps, with the schedule for each step set by appropriate legislation.

Each state shall appoint an electoral committee for the next term in the same manner as it appoints its senators. Each electoral committee shall compile an ordered list of the most preferred to least preferred candidate. Such ordering may be constrained by state law. These committees shall transmit their respective lists as ballots to the president of the senate.  The President of the Senate shall, in the presence of the Senate and House of Representatives, open all the certificates and read them publicly. The votes of the committees shall then be tabulated.
\end{quoting}

In the multi-senator version of this proposal, a state's ballot could be counted multiple times. States might use popular elections to inform and constrain the committees.

\begin{quoting}
A ballot is considered to prefer one candidate over another when the former candidate appears higher on the list than the latter candidate, or if the former candidate appears on the list while the latter candidate is absent.

One candidate is considered preferred to another candidate as a whole if the number of ballots that prefer the former candidate to the latter is greater than the number of ballots that prefer the latter to the former.

Should there exist a candidate that is preferred to every other candidate by the ballot of the electoral committees, this individual is considered a clear winner and becomes the next councilor.
\end{quoting}

The electoral system here is looking for a Condorcet winner, named after an 18th century French mathematician and politician who did foundational work in social choice theory. This winner may not exist, but if it does this person would be the best choice for the group.

The reason to use a Condorcet method here is that it encourages and ultimately selects strong independent candidates in a partisan environment. For example, suppose that 40\% of the committees prefer the partisan democrat, 40\% prefer the partisan republican, and 20\% support a strong independent candidate. Under instant runoff voting, that independent candidate would be eliminated in the first round. Under the Condorcet method, the independent could win because we can imagine both the Democrats and the Republicans prefer them to the other side.

The tabulation in this system is a bit more complex than alternative ranked-choice methods like instant runoff voting. It also requires the electoral committees to have researched and ordered all the candidates. What may not be a workable system in a large-scale election becomes possible with the electoral committees. The orderings could come from the result of some state-wide election, or it could come from the state legislature or the committees themselves.

\begin{quoting}
Should no such candidate exist, the tabulators shall then determine the smallest group of candidates who are preferred to every candidate outside the group. The Senate shall then decide among these candidates unless there is a law that provides a formula for choosing among these candidates from the ballot rankings.
\end{quoting}

If there is no winner, this means that there is either a tie or a cycle of preferences. If this is the case, we then look for the “Smith set”\cite{Smith_Set}, which is the set of potential winners. There are many existing ways to choose among this set, with names like the Schulze method\cite{Schulze} and minimax, but it is still unclear which would lead to the best result in this case. Therefore, we can allow this decision to be changed through legislation. Note that the law can only determine the outcome based on information from the committee lists and not on other characteristics such as “who is older” or “who won a coin toss”.

By default, we have a contingency election in the Senate to ensure a final decision. With these elections occurring annually, data will begin to accumulate to inform election law.

\begin{quoting}
When there is a vacancy on the council, the council shall fill that vacancy by setting a date for the electoral committees from that term to present new lists to the Senate.
\end{quoting}

Why do we want the states to appoint electoral committees and not just compile the ranking directly? It is because these electors will be consulted on any vacancies that occur. It’s a way to respect the 6 year term and disincentivize removals from office of council members for political reasons, as the same electoral committees will choose the replacement. It also allows a councilor to resign while minimizing some vast partisan swing.

\subsection{Powers of the Executive Council}

\begin{quoting}
With a majority vote, the Council may recommend a repeal of legislation signed into law no less than six years prior. With such a recommendation, both houses of congress have 90 days to confirm the legislation with a majority vote or the repeal is effective.
\end{quoting}

The next clause echoes article 56 of the New Hampshire constitution\cite{New Hampshire Constitution} for ensuring a sound fiscal policy.

\begin{quoting}
No money shall be issued out of the treasury (except as may be appropriated for the redemption of bills of credit, or treasurer´s notes, or for the payment of interest) but by the president with advice and consent from the council that this disbursement is consistent with Article 1 Section 9 pertaining to treasury disbursements.

The Executive Council shall inherit from the previously constituted senate the role of providing advice and consent to presidential appointments.

The Executive Council shall have the power to nominate justices of the Supreme Court, as well as ambassadors and other public ministers, subject to the advice and consent of the Senate.
\end{quoting}

The crucial impact that Supreme court appointments have on constitutional law suggests that it ought to be debated by an adversarial body like the senate. However, the council seems to be a more fitting institution for the nominations than the president\footnote{An imperfect though amusing analogy would be the NBA allowing the national champions to replace retiring referees each year. They might not screw it up, but they are not the appropriate committee.}. For other appointments, the senate already gives the president a lot of leeway. The executive council would be better positioned to provide the “advice” part of this task, with their long terms and national constituency.

\begin{quoting}
A council vote attains a supermajority if at least three fifths of the members vote in the affirmative, or equivalently 5 members from a full council of 7.
\end{quoting}

The following are very serious actions, and therefore we require the legitimacy of a supermajority. This provides a more moderate approach to an after-the-fact line item veto and a way to fight against unconstitutional legislation.

\begin{quoting}
With a supermajority the council may reduce the amount of any appropriation made by law, so long as the disbursements from that appropriation are under their power of review by law\footnote{They should not be able to block interest payments and other government promises}.

With a supermajority, the council may declare any legislation unconstitutional or a threat to the constitutional system of government, and shall include their reasons for any such declaration. The legislation is immediately null and void. However, any member of congress has standing to challenge this reasoning at the supreme court which has the authority to evaluate and overrule the decision.
\end{quoting}

\subsection{Reining in the Administrative State}

There is growing recognition that the unelected administrative state is not only an affront to the values that the American government was supposed to represent, but that it has been injurious to individual liberty as well. Examples of these agencies include the SEC\footnote{In SEC v LBRY, the Securities and Exchange Commision successfully shut down\cite{Barash} an innovative and consumer-focused company in a space rife with scams, fraud, and get rich quick schemes. After many years, the SEC has refused to clarify the specific rules that were broken and how to come into compliance. At the same time this is going on, the SEC also failed to protect consumers from such frauds.}, the EPA, the FDA, the FCC, the FTC, the IRS, the Bureau of Land Management, the CDC, and OSHA of vaccine mandate fame. Books and volumes could be assembled documenting the abuse of power and appearance thereof within these agencies. That is not to say that the officers or employees of these agencies are all somehow corrupt or unaccountable, but that their accountability is not properly weighted towards the people that they are supposed to serve or even the people's elected representatives.

In a sense, congress has outsourced some of its legislative power to these agencies by empowering them to make specialized rules. As executive agencies they additionally have the power to choose who to target for enforcement and when to look the other way. The political appointees of these agencies - often with the Commissioner title - have split accountability being nominated by the President and approved by the Senate. Some of the terms for these commissions last many years! And under these political appointees are policy-making officials who are part of the permanent bureaucracy.

Similarly situated are the intelligence agencies: the CIA, FBI, and NSA. Not only do these organizations suffer from the same issues as the administrative state, but they seem to use their positions to unduly influence elections and elected officials. According to Senate majority leader Chuck Schumer\cite{Stanley}, "you take on the intelligence community, they have six ways from Sunday at getting back at you." These agencies systematically seek to censor speech by claiming concern over misinformation and foreign propaganda\cite{Beanz}.

Recent efforts to rein this in have come from the judiciary\cite{Crowell} and the presidency\cite{ScheduleF}. The judicial process takes years to resolve while presidential efforts have been stalled since the 2020 election.

How did we get here? It was the structure of government that has allowed this to happen in the first place. The new Executive Council is in a better position than any existing institution to check the increasing power of the unelected administrative state and intelligence agencies.

The President can currently remove some appointments but would have a more difficult time changing policy directly\cite{Fairlie}. Perhaps that ability could be given directly to the President, but it would be unlikely for a single individual to wield that power in a thoughtful way that confers legitimacy. Therefore, we give the president the ability to overturn administrative policy with the consent of the council, and if the council is nearly unanimous they can do it themselves.

\begin{quoting}
With either a supermajority or a simple majority that includes the president, the council may alter or nullify any administrative policy originating in the executive branch. They may also remove administrative officers and employees who make policy. This clause does not remove any powers previously held by the President to act unilaterally in cases of policy and removal.
\end{quoting}

The rule from a supermajority without the president is designed to give the council a moderating pushback on presidential actions. The bar is set high to prevent the appearance of a continual coup over the President from the Council. More input from experts is needed in order to ensure that the application of this clause has the desired coverage and to determine what counts as an administrative policy or a policy-making employee.

\subsection{Impeachment and Compensation}

\begin{quoting}
Executive councilors are subject to the federal impeachment process.

Executive councilors shall be well compensated as set out by law.
\end{quoting}

One could argue that decisions by the Executive Council might be just as bad as those by the administrations that we have today. This is unlikely because of the incentive structure of the council, their position within the government, and their manner of election.

For example: why should the federal government provide Councilors with a high-level executive compensation package? It is to ensure a strong pool of talent from many sectors of public and private life for any councilor election; and this strong talent pool will lead to better fiscal management. Furthermore, councilors are not beholden to any one department or party. And in a committee of seven, a dissenting voice cannot be ignored. Do not expect a heavily ideological group apart from the President. If the system works well, we might attain talent from business, foreign affairs, politics, and other professions. A group of long-term strategic thinkers is more likely to take the bill of rights seriously.

\section{Electing the President}

We start here by removing the anachronistic “natural born citizen” clause which the founders included to prevent European royalty from coming over and capturing the government. The phrasing comes from Orin Hatch\cite{Somin_Hatch}, with a reduction of the 20 year requirement as convincingly argued by Ilya Somin\cite{Somin_NBC}.

\begin{quoting}
\textbf{Section 4}

A person who has been a citizen of the United States for 7 years, is not ineligible to hold office as President of the United States by reason of not being a native-born citizen.
\end{quoting}

The electoral college is currently proportioned by adding the number of senate and house seats from each state. That formula depends entirely on the size of the house with respect to the senate which was selected for entirely different reasons than to affect presidential elector apportionment.

Now that we have a stronger executive council whose electors are apportioned like the senate, we can drop the senate-based apportionment term from the presidential electoral college base it purely on population.

\begin{quoting}
The number of electors for president shall be equal to the number of representatives to which the state and territory may be entitled, and shall be elected in the same manner as its representatives. If the electoral college fails to yield a majority vote for any candidate and a contingency election occurs in the house of representatives, that contingency election shall apportion one vote per representative.
\end{quoting}

The pattern is now complete. The president will be chosen on the same basis as the house of representatives, and the executive council will be chosen on the same basis as the senate. Both DC and the US territories will have representation in the electoral college. This supersedes the 23rd amendment.

This system removes one criticism of the electoral college, that regions are not represented uniformly based on their population. However, it does not create a national popular vote. Organizing voters into districts and constituencies is still valuable in its fault tolerance. In other words, it limits the effect of local anomalies.

There is a connection here with the recent interest in neural network architecture. These networks are organized into layers which are each filtered through an activation function which truncates one or both sides of its input. This architecture has proven successful in training intelligent software. Because such networks are often found in nature, we can consider our electoral college not a shameful relic of the past, but as the neural network of democracy.

Of course, these ideas will be debated and revisited many times in the future. This is why this proposal encourages a vigorous debate over election law in the courtrooms and the marketplace of ideas.

\section{Summary}

In table~\ref{table:institutions}, we summarize the four institutions laid out in this document and their various properties. We have also sketched out the powers that each branch should have. More feedback from across the political spectrum is needed in order to achieve the broad-based balance as intended.

\begin{table}[ht]
\centering
\renewcommand{\arraystretch}{1.5}
\begin{tabular}{|c|c|c|c|c|}
\hline
\textbf{Institution} & \makecell{House of \\ Representatives} & Senate & Presidency & \makecell{Executive \\ Council}  \\
\hline
\textbf{Purpose} & Voice of the People & Voice of the States & Leadership & \makecell{Professionalism \\ Continuity} \\
\hline
\textbf{Branch} & Legislative & Legislative & Executive & Executive \\
\hline
\textbf{Term Length} & 2 Years & State Law & 4 Years & 6 Years \\
\hline
\textbf{Process} & Adversarial & Adversarial & Unitary & Cooperative \\
\hline
\textbf{Apportionment} & Population & States & Population & States  \\
\hline
\textbf{Electors} & Direct per District & State Law  & Electoral College & Electoral College  \\
\hline
\textbf{Winning Condition} & Plurality & State Law & Majority & Condorcet  \\
\hline
% you can continue adding rows in the same format
\end{tabular}
\caption{A summary of the proposed legislative and executive institutions and their properties. }
\label{table:institutions}
\end{table}

\section*{Acknowledgements}

I was fortunate to have Aaron Bell, my longtime friend and collaborator on The Local Maximum podcast, helped me hash out the provisions of this proposed amendment and fixed its initial shortcomings. Our conversations can be heard in episodes 283 through 285 of The Local Maximum, as well as various discussions on voting systems, social choice theory, and constitutions that we've had over the years. I was also fortunate to have Jordan Marks, Tax Assessor for San Diego County, provide his commentary and to encourage me to think carefully and systematically about the potentials and pitfalls to having various boards and commissions in government.

\appendix
\section{Supplementing the House with Representative Assemblies}
\label{appendix:house}

The following is a proposal (which can be brought about through legislation or amendment) to support the House of Representatives with creating representative assemblies around the country that can truly represent the people. Assembly members are not required to familiarize themselves with legislation. They are there to hold their own representatives accountable as citizens and to occasionally decide on referendums.

\begin{quoting}
Each house district shall have a representative assembly with terms coinciding with the term for the House of Representatives. The size of each representative assembly shall be proportional to the population of that district. Elections for each assembly shall occur at-large with seats awarded proportionally to the elected preferences of the voters.

Assemblies have the power to elect and recall their representative, to arrange sessions for their representative to answer questions from the assembly, to elect a new representative in case of a vacancy, and to participate in assembly referendums.
\end{quoting}

This is not supposed to be an “indirect election” to the house. Voters will still know which house candidate they are voting for, with the added benefit of their preference being represented in their local assembly. Note the two terms of proportionality here. The first ensures that the number of assembly members per capita is about even throughout the country. The second calls for a specific type of electoral system - perhaps party-list proportional representation - that can more easily represent minority voices.

\begin{quoting}
Members of representative assemblies shall be volunteers, only receiving reimbursement funds related to assembly business. Assemblies shall keep a light schedule and make reasonable accommodations for members in various life circumstances. Representative assemblies shall meet in public in their home district except when due to unusual conditions.

Assembly directives are rules governing the House of Representatives and the assemblies themselves. Assembly directives are voted upon in an assembly referendum. During a referendum, all assemblies including those for non-voting members meet to debate one or more resolutions and vote on them. The vote totals are transmitted to the House to tabulate the official total and determine if a majority of assembly members nationally have voted in favor.
\end{quoting}

Because the assemblies are very evenly apportioned by population and the proportional nature of their election cannot easily be manipulated by redistricting, this referendum is truly democratic in a way the house can only approximate. The idea here is that the assemblies can govern the house, particularly on questions of procedure and ethics, to ensure that it represents the people as best as possible. Note that article 1 section 5 states that the House should determine its own proceedings, so this change may require a constitutional amendment.

\begin{quoting}
The internal rules for the House of Representatives and assemblies shall be decided by the following in order of precedence: the constitution, assembly directives, and each body’s own rules. Additionally, the house may make rules for the assemblies that override each assembly’s own rules but will not override assembly directives.
\end{quoting}

This document does not and should not decide on all the rules for governance of these assemblies. The initial rules would come from the house itself, which could then be codified into directives. These initial rules might include: providing a formula for the size of the assemblies, the proportional voting system for assemblies, how assemblies choose their representative, how to break deadlocks, requiring a supermajority for an assembly to remove their representative, and how to trigger a referendum.

While this idea seems to put into practice the democratic nature of our government that was promised, there are many open questions about its feasibility and necessity. I hope it will provoke discussion just as the rest of the proposal.


\begin{thebibliography}{20}
\bibitem{Tucker}Tucker, Jeffrey. \href{https://www.theepochtimes.com/repeal-the-17th-amendment-now\_4909126.html}{“Repeal the 17th Amendment Now.”} The Epoch Times, December 7, 2022.

\bibitem{Virginia}McInerney, Virginia. \href{https://lonang.com/commentaries/conlaw/federalism/repeal-seventeenth-amendment/}{“Why the 17th Amendment Needs to Be Repealed.”} LONANG Institute, March 27, 2020.

\bibitem{Smith}Smith, Terry. \href{https://www.latimes.com/archives/la-xpm-2010-oct-22-la-oew-smith-17th-amendment-20101022-story.html}{“Why We Have, and Should Keep, the 17th Amendment.”} Los Angeles Times, October 22, 2010.

\bibitem{Komoroske}Komoroske, Alex. \href{https://medium.com/@komorama/on-schelling-points-in-organizations-e90647cdd81b}{“On Schelling Points in Organizations.”} Medium, December 31, 2021.

\bibitem{Allen_Milligan}Allen v. Milligan, 599 U.S. (2023)

\bibitem{Brewer}Brewer, Simon. \href{https://law.yale.edu/mfia/case-disclosed/back-basics-why-partisan-gerrymandering-violates-first-amendment. }{“Back to Basics: Why Partisan Gerrymandering Violates the First Amendment.”} Yale Law School, Media Freedom \& Information Access Clinic, March 12, 2019.

\bibitem{Allen}Allen, Danielle. “Opinion | \href{https://www.washingtonpost.com/opinions/2023/03/28/danielle-allen-democracy-reform-house-representatives-districts/}{Just How Big Should the House Be? Let’s Do the Math}.” The Washington Post, May 23, 2023. 

\bibitem{30000}“\href{https://thirty-thousand.org/the-house-of-representatives-is-scalable/}{Learn Why a Larger House Will Be Much More Productive and Enable the Representatives to Better Serve Their Constituents}.” thirty-thousand.org, June 6, 2022.

\bibitem{Federalist10}Madison, James. ”The Federalist Papers, No. 10.” November 22, 1787 (1787).

\bibitem{Federalist62}Madison, James. ”The Federalist Papers, No. 62.” Independent Journal, February 27 (1788).

\bibitem{Eisinger}Eisinger, Vince. ”Auxiliary Protections: Why the Founders' Bicameral Congress Depended on Senators Elected by State Legislatures.” Touro L. Rev. 31 (2014): 231.

\bibitem{Schiller}Schiller, Wendy J., and Charles Stewart III. \href{https://www.brookings.edu/wp-content/uploads/2016/06/Schiller_17th-Amendment_v7.pdf}{”The 100th Anniversary of the 17th Amendment: A Promise Unfulfilled?”}. Issues in Governance Studies at the Brookings Institute. May, 2013

\bibitem{Madison}Madison, James. Original Notes on Debates in the Congress of Confederation, 1787.

\bibitem{Senate}\href{https://www.senate.gov/about/origins-foundations/senate-and-constitution.htm}{“About the Senate and the Constitution.”} U.S. Senate: About the Senate and the Constitution, September 1, 2022.

\bibitem{Amar}Amar, Akhil Reed. ”Philadelphia Revisited: Amending the Constitution Outside Article V.” U. Chi. L. Rev. 55 (1988): 1043.

\bibitem{Shapley}Shapley, Lloyd S., and Martin Shubik. "A method for evaluating the distribution of power in a committee system." American political science review 48, no. 3 (1954): 787-792.

\bibitem{Gross}Gross, Benjamin R. "Solving the Electoral College." PhD diss., Colorado College., 2010.

\bibitem{Lublin}Lublin, Joann S. \href{https://www.wsj.com/articles/smaller-boards-get-bigger-returns-1409078628}{“Smaller Boards Get Bigger Returns.”} The Wall Street Journal, August 29, 2014. 

\bibitem{Khanna}Khanna, Parag. \href{https://qz.com/876260/seven-presidents-are-better-than-one-why-the-oval-office-needs-a-round-table}{“Seven Presidents Are Better than One: Why the Oval Office Needs a Round Table.”} Quartz, January 10, 2017.

\bibitem{Hahn-Burkett}Hahn-Burkett, Tracy. \href{https://www.concordmonitor.com/What-is-the-Executive-Council-34817477}{“3-Minute Civics: What Is the Executive Council?”} Concord Monitor, June 23, 2020.

\bibitem{Timmins}Timmins, Annmarie. \href{https://www.nhpr.org/nh-news/2021-10-25/executive-council}{“The Behind-the-Scenes Power of N.H.’s Executive Council Is Now at Center Stage.”} New Hampshire Public Radio, October 25, 2021. 

\bibitem{Ruger}Ruger, William, and Jason Sorens. Freedom in the 50 states, 2013 edition: An index of personal and economic freedom. The Mercatus Center at George Mason University, 2013.

\bibitem{Kaufman}Kaufmann, Bruno. “\href{https://www.swissinfo.ch/eng/business/the-strengths-of-a--weak--swiss-government/48483858. }{The Strengths of a ‘weak’ Swiss Government}.” SWI swissinfo.ch, May 5, 2023. 

\bibitem{Purnick}Purnick, Joyce. “\href{https://www.nytimes.com/1986/02/16/weekinreview/the-board-of-estimate-and-its-critics.html}{The Board of Estimate and Its Critics}.” The New York Times, February 16, 1986.

\bibitem{Board_of_Estimate}Board of Estimate of NYC v. Morris, 489 U.S. 688 (1989)

\bibitem{Mason}Rutland, Robert. ”The Papers of George Mason”, 3 vols. The University of North Carolina Press. Chapel Hill, NC. 1970.

\bibitem{Mason_Objection}Mason, George. ”Objections to the Constitution of Government Formed by the Convention.” Allen and Lloyd, The Essential Antifederalist (1787): 12.

\bibitem{Cooper}Cooper, Charles J. \href{https://www.nationalaffairs.com/publications/detail/confronting-the-administrative-state}{“Confronting the Administrative State.”} National Affairs, Fall 2015.

\bibitem{Yang}Yang, Stephanie, and Max Sklar. "Detecting Trending Venues Using Foursquare's Data." In RecSys Posters. 2016.

\bibitem{Smith_Set}Smith, John H. "Aggregation of preferences with variable electorate." Econometrica: Journal of the Econometric Society (1973): 1027-1041.

\bibitem{Schulze}Schulze, Markus. ”The Schulze method of voting.” arXiv preprint arXiv:1804.02973 (2018).

\bibitem{New Hampshire Constitution}Constitution of New Hampshire, Part 2, Article 56

\bibitem{Barash}Barash, Martina. \href{https://news.bloomberglaw.com/ip-law/lbry-loses-crypto-token-fight-with-sec-but-vows-it-wont-appeal}{“LBRY Loses Crypto Fight with SEC, Says It Will Shut down (1).”} Bloomberg Law, July 12, 2023.

\bibitem{Stanley}Stanley, Jay. \href{https://www.aclu.org/news/national-security/do-us-politicians-need-fear-our-intelligence}{“Do U.S. Politicians Need to Fear Our Intelligence Agencies?: ACLU.”} American Civil Liberties Union, February 27, 2023.

\bibitem{Beanz}Beanz, Tracy. \href{https://www.uncoverdc.com/2023/05/24/the-lead-up-to-the-hearing-missouri-v-biden/}{“The Lead up to the Hearing: Missouri v. Biden.”} UncoverDC, May 25, 2023. 

\bibitem{Crowell}”\href{https://www.crowell.com/a/web/bomv5ATK9LZPNrBA51skWq/4TtiyY/Regulatory-Forecast-2020-Administrative-Law-Crowell-Moring.pdf }{Administrative Law – The Supreme Court and the President Rein in the `Administrative State`}.” Regulatory Forecast 2020, 2020. 

\bibitem{ScheduleF}Executive Order No. 13957. (2020) \href{https://www.
federalregister.gov/documents/2020/10/26/2020-23780/creating-schedule-f-in-the-excepted-service}{Creating schedule F in the excepted service}, October 21, 2020.

\bibitem{Fairlie}Fairlie, John A. ”The Administrative Powers of the President.” Michigan Law Review (1903): 190-210.

\bibitem{Somin_Hatch}Somin, Ilya. \href{https://reason.com/volokh/2020/08/16/orrin-hatchs-constitutional-amendment-to-abolish-the-natural-born-citizen-clause/}{“Orrin Hatch’s Constitutional Amendment to Abolish the Natural Born Citizen Clause.”} Reason.com, August 29, 2020.

\bibitem{Somin_NBC}Somin, Ilya. \href{https://reason.com/volokh/2020/08/14/why-we-should-abolish-the-requirement-that-the-president-must-be-a-natural-born-citizen/}{“Why We Should Abolish the Requirement That the President Must Be a ‘Natural Born Citizen’ [Updated].”} Reason.com, August 29, 2020. 

\end{thebibliography}

This document along with revisions is posted at github as https://github.com/maxsklar/great-compromise-ammendment. See readme for contact information. No funds, grants, or other support was received.
\end{document}
