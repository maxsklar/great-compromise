\documentclass{article}
\usepackage{amssymb}
\usepackage{amsmath}
\usepackage{algorithm}
\usepackage{algpseudocode}
\usepackage{pgfplots}
\usepackage{multicol}
\usepackage[hmarginratio=1:1,top=32mm,columnsep=20pt]{geometry}
\usepackage{fullpage}
\usepackage{pdflscape}
\usepackage[toc,page]{appendix}
\usepackage[shortlabels]{enumitem}
\usepackage[font=itshape]{quoting}
\usepackage{tabularx} % for 'tabularx' environment and 'X' column type
\usepackage{makecell}
\usepackage[hidelinks]{hyperref}
\usepackage{setspace}

\newcommand{\quotes}[1]{``#1''}

\begin{document}
\parindent=0in
\parskip=12pt

\title{
  Toward a New Great Compromise - DRAFT \\
  \large{
    A Constitutional Plan to Promote Liberty, Renew Democracy, and Restore Federalism
  }
}

\author{Max Sklar, Local Maximum Labs, August, 2023}
\date{}

\maketitle

In 1787, the delegates to the Constitutional Convention arrived in Philadelphia with a rare and historic task: to debate, draft, and negotiate their way towards a document that would structure the government of the United States and become the supreme law of the land. To this end, the \quotes{Great Compromise} was forged which established the American governing institutions that we know today. Despite the Convention's success in creating an enduring structure, they could not escape the fact that the contradictions in such a system tend to become more pronounced over time. At certain points, these contradictions must be resolved one way or another. The most glaring example of slavery was surely known to the founding generation and infamously came to a head when no voluntary resolution arrived.

Any contradictions in the Constitution today seem more subtle in comparison. They appear more technical than moral, and relate more to checks on power and apportionment. But they still exist and are widening with evermore profound effects on our rights and wellbeing. If well-meaning Americans fail to deliberate how the government's basic structure, one day these decisions will be left to the hands of nefarious actors. Our approach should neither deify nor dismiss the constitution's original framers. We should instead seek to understand why they made those critical decisions as they did, and how those decisions have echoed throughout the course of history. It is through this understanding that we can evaluate future changes.

\section{The Emerging Crisis}

Many Americans fear their liberties are under attack and liberal democracy is slipping away. Because of the extremely polarized environment, they cannot even agree on the source of the problem. This fracture partially sets centralized urban populations against widely distributed rural ones. Neither side is going away, and despite misguided hopes and promises neither has enough support to permanently impose their will on the other. In the spirit of the 1787 Compromise that founded this country, we seek a principled plan that has properties attractive to both the left and the right, and that could garner broad support after public debate.

Furthermore, this plan seeks to uphold the historic principles upon which the American government rests. First is that it should be a federal republic with checks and balances to maintain the rule of law. Another is that it should be democratic both in how it derives its power through its citizens and in its approach to transparency and open debate. Of critical importance is the ability for multiple parties to compete for office and respect for individual liberties as famously codified in the Bill of Rights.

Every conversation for great improvement must begin somewhere. I humbly seek your feedback for improvements.

\section{A New Great Compromise}

The rural faction would be well served by a repeal of the 17th amendment\cite{Tucker}\cite{Virginia} which required the direct election of senators. With its repeal, the senate would be selected by the state legislatures. This would presumably lead to a less \quotes{top heavy} federal structure with states having more control over local issues. However, the progressive faction would undoubtedly defend the 17th amendment, as they have benefited from urban votes and are skeptical towards indirect elections\cite{Smith}. They would also be concerned about state legislatures being elected on the basis of unfair districting schemes, or gerrymandering. Even with fair district maps, the impact of urban machines is capped because each district can only elect a single representative. One might counter by questioning the legitimacy or wisdom of the techniques used for stacking this votes in dense regions such as ballot harvesting.

Instead of repealing the 17th amendment directly, this proposal seeks a comprehensive change to the structure of government that is both principled and satisfactory to the various factions. It includes better representation for the state governments than a repeal of the 17th could ever hope to achieve, along with redistricting and apportionment reforms long sought after by the progressive faction. Furthermore, an executive council will be created to solve long-standing shortcomings of the current structure, which includes taking back a degree of control and accountability from the permanent bureaucracy.

Institutions degrade over time as they attract special interests and ideologues who would seek to respurpose them. To combat this, institutions should have a clear and credible purpose. This provides a schelling point, or a default manner in decision-making, around which its members can agree as a first principle\cite{Komoroske}. 

The Constitution's framers had competing ideas for the Senate: was going to be the \quotes{voice of the states} or the \quotes{more professional and permanent upper chamber} They ended up combining both. Our working theory is that this arrangement backfired, preventing the Senate from optimally serving either purpose. The resulting confusion eventually leading to the 17th amendment.

By separating out these concerns into a Senate and an Execute Council, we can have a government with democracy, federalism, and continuity. As will be demonstrated, there is quite a bit of support for this structure historically. It can be found in successful world and state governments, the private sector, and even from the framers themselves.

The changes are summarized as follows:
\begin{enumerate}
  \item The \textbf{House of Representatives} will include representation for the US territories and DC along with a stronger judicial oversight to evaluate election law and district maps.
  \item Members of the \textbf{Senate} are chosen by their respective states through state law and are in office at-will, thereby establishing the accountability they currently lack.
  \item An \textbf{Executive Council} will be established to replace aspects of the Senate that require long term lengths. This council will provide much needed oversight and accountability to the permanent bureaucracy. In order to encourage independent candidates for councilor, the first national Condorcet election will be introduced.
  \item The Electoral College for the \textbf{President} will be apportioned by population like the House of Representatives.
 \item Each institution is given a clear purpose and mission.
\end{enumerate}

The rest of this document explains the reasons for these proposals though the drafting of a constitutional amendment.

\section{The House of Representatives}

\begin{quoting}
\textbf{Section 1}

The purpose of the House of Representatives is to represent in its members the broad spectrum of the people of the United States in the legislative process.
\end{quoting}

Come election time, representatives must at least theoretically answer to their constituents for most of the decisions they have made. This is less true when it comes to legislating those elections themselves. They are literally changing their constituents! The courts have therefore taken a role in evaluating election law, but continue to grapple with how to define that role. While the courts have found a legal basis to rule against racial gerrymandering, they have struggled to combat partisan gerrymandering\cite{Brewer}. We seek to expand the role of the courts here.

\begin{quoting}
The manner in which states elect their representatives shall be designed to produce a delegation that is broadly representative of the population of that state and its various communities given the delegation size. This manner of election falls first to the states, and then to federal law which override such regulations.

Any group of electors shall have standing to bring suit in the federal judiciary with claims that these election regulations are misaligned with their constitutional purpose. In considering such cases, the judiciary shall take into account the totality of
\begin{itemize}
\item the historical precedent,
\item the changing facts, particularly concerning the electorate, which may degrade a once-valid electoral system,
\item the empirical properties of electoral law as discovered through observation of elections,
\item the theoretical properties of electoral law as discovered through logical deduction.
\end{itemize}
\end{quoting}

This finally gives courts a mandate and a roadmap to evaluate questions of redistricting. Electoral system design is a complex problem, and no magic solution exists to achieve mathematical perfection. Some might say that an abolition of districts altogether in favor of a proportional system would eliminate these problems, but there are pitfalls to such a system as well.

Rather than claiming to have found the best rules, we can instead set ourselves up to evolve with a north star statement like: \quotes{broadly representative of the population of that state and its various communities}. Judges will grapple with the idea of \quotes{various communities} as each case arises, and will come to a set of principles for evaluating claims of misrepresentation over time.

With that established, it makes sense to ensure that all citizens get a voice in the House of Representatives, including those who do not reside in any state.

\begin{quoting}
Seats in the House of Representatives shall be apportioned for the District of Columbia and United States territories. The seats allocated to each entity shall be based on its population just the same as if it were a state. Territories with populations less than half of the number of house seats per capita shall instead receive a non-voting member.
\end{quoting}

This provides approximately 1 seat for DC and 3 seats for Puerto Rico, with the smaller territories receiving non-voting members as is currently the case.

Congress should consider increasing the size of the House in order to better represent the large population of the United States. Several plans to do this have been developed, ranging from a modest increase to a gargantuan one\cite{Allen}. A plan provided by thiry-thousand.org even calls for around 8000 representatives in order to satisfy the low population-to-representative ratio envisioned at constitutional convention\cite{30000}!

Madison urged caution with this in Federalist 10 when he writes, \quotes{the representatives must be raised to a certain number, in order to guard against the cabals of a few... they must be limited to a certain number, in order to guard against the confusion of a multitude.}\cite{Federalist10} With that in mind, appendix~\ref{appendix:house} provides a sketch of how the House could be restructured with more voices without dramatically increasing its size.

\section{Senate}
\label{section:Senate}

The constitutional convention was split between the Senate existing to \quotes{represent the states} and to be a \quotes{long-term, more professional, upper house}. Hamilton and Madison saw it as the latter, with Hamilton originally proposing lifetime terms, and Madison originally proposing that they be appointed by the house.

Even the delegates who proposed senators be elected by the state legislatures were not motivated entirely by concern state agency. John Dickinson \quotes{wished the Senate to consist of the most distinguished characters, distinguished for their rank in life and their weight of property, and bearing as strong a likeness to the British House of Lords as possible; and he thought such characters more likely to be selected by the State Legislatures, than in any other mode.}\cite{Madison}

Dickinson's system prevailed at the convention and Madison ultimately promoted it Federalist 62 when he wrote, \quotes{It is recommended by the double advantage of favoring a select appointment, and of giving to the State governments such an agency in the formation of the federal government as must secure the authority of the former, and may form a convenient link between the two systems.}\cite{Federalist62}

Both Madison and Dickinson cared about securing a degree of \quotes{agency} for the states, but they often presented it as a secondary benefit of their design. With these competing conceptions of the Senate, the framers never gave a consistent message. It is no wonder that the 17th amendment was ratified without much debate around this!

The 17th amendment was passed by in 1912 with the hopes that direct elections of senators would lead to more democratic accountability\cite{Eisinger}. Has it delivered on its promise? A report by the Brookings Institute on the 100th anniversary of the 17th amendment found that none of this accountability could be verified by political data\cite{Schiller}. Senators elected directly seem to be no more responsive to voters than senators elected indirectly.

It actually makes sense: Why should they be? They have a six year term. Their constituents' voting power is diluted among millions, and they can more easily win elections by courting doners and corporate media than by listening to individual concerns\footnote{As the character Pappy O'Daniel said in Oh Brother, Where art Thou, \quotes{We ain't one-at-a-timin' here. We're MASS communicatin'!}}. The progressive authors of the 17th amendment mistakenly believed that it was the indirect election of the senators that made them unaccountable without considering their term length.

The convention delegates on the other hand understood the term length tradeoff; that longer terms led to independence and shorter terms led to accountability. Roger Sherman, Connecticut delegate and author of the \quotes{Great Compromise} warned against long term lengths when he said, \quotes{The more permanency it has the worse if it be a bad government.}\cite{Madison}

To ameliorate this confusion around the Senate's purpose, we separate it into 2 parts: the new Senate will exist entirely to represent the states. Its members are now subject to at-will employment. Then we will create an Executive Council to make up for what this new Senate now lacks.

\begin{quoting}
\textbf{Section 2}

The purpose of the Senate is to represent the interests of the individual states in the legislative process.

Each state shall appoint one senator, and may appoint any number of alternate senators. Such alternates may engage in Senate business in the senator’s absence, according to a clear order of succession indicated by their constituent state.
\end{quoting}

This section begins with a change to Article 1 Section 3 on the composition of the Senate and overrides the 17th amendment. Some senators today exist only to provide party-line votes without any talent as a legislator. Others are political relics who can coast from familiarity with voters and donors. These types would find it difficult to last long in the new Senate, whose members would need to closely monitor the effects of federal legislation on their particular state. As a result, federal laws would not encroach on the states as much as they currently do. Unfunded mandates would come under particular scrutiny.

The constitution assigns two senators per state to prevent absences from leaving a state unrepresented. Today, ease of communication and travel make absences less necessary during key votes. Still, Senators will not always be available, so we will allow alternates. Convention delegate Martin Luther from Maryland called for 1 vote per state, as it was under the Continental Congress and the Articles of Confederation\cite{Senate}.

Interestingly, this plan does not rely on the equal suffrage of states in the Senate. We keep the equal suffrage to remain consistent with Article V which renders this topic is off limits in the amendment process\cite{Amar}. However, some reapportionment in order to satisfy the progressive faction would not be misaligned with the Senate's stated purpose. A modest plan could allocate half the senate equally by state and the other half by population. This can be supported in principle because mathematical analysis of electoral systems, such as the one developed by Shapley and Shubik\cite{Shapley}, show that block voting is worth more than the sum of its parts\cite{Gross} and to give large states enormous blocks of vote - even if proportional to their population - would give them more than that proportion of decision-making power.

\begin{quoting}
The method of selecting and removing senators shall be determined by the laws of the state they represent. A state may appoint, recall, or replace their senate delegation at any time in accordance with the laws of that state. The senate shall determine reasonable and orderly procedures for updating its roster and handling rotations as it receives notices of appointment, recall, and replacement from the states.

Absent relevant state laws, senators shall be selected by the state legislature. If the state legislature does not select a senator, the highest executive officer in the state can appoint a senator to serve in the interim. This method of appointment is overruled only by state laws passed after this amendment is ratified.
\end{quoting}

The authors of the 17th amendment were concerned about deadlocks that occurred where no senator was chosen. This solves it. Unlike in the original constitution, the method of choosing senators is determined by state law. The states would undoubtedly come up with a variety of systems instead of relying on the default.

We next change Article 1 Section 6 concerning the salaries of senators.

\begin{quoting}
Senators and alternate senators shall not receive any compensation from the treasury of the United States. They may receive compensation from the state they represent, as prescribed by the laws of that constituent state.
\end{quoting}

Madison wanted senators to serve for 6 years in order to provide stability and continuity in government.\cite{Senate} Because of this, salaries needed to come from the federal government so that a state could not effectively recall a senator (or exert undue influence) by halting payments. Now that the senator serves at the pleasure of their state, that senator can get paid by the state.

\begin{quoting}
The Senate shall retain the power to provide advice and consent on the appointments of justices of the Supreme Court, ambassadors, and other public ministers. They also retain the power to approve treaties, to try impeachments, and as a legislative body.
\end{quoting}

\section{The Executive Council}

Today we would be hard pressed to find support for a \quotes{House of Lords} as some founders prescribed.

Instead, we can recognize the value in having certain long-termed elected positions. It allows these officeholders to focus on the job at hand rather than their reelection campaign, and to pursue long term goals as required to preserve and expand the assets of the county. It also provides a sense of continuity and stability to allow for institutional knowledge to be retained.

The Senate was originally tasked with this role. The new Senate may have some members with long tenure, but they will be continually held accountable to their state. So instead we create the Executive Council and place it in the executive branch to fulfill this role. The Executive Council will be modeled as a \quotes{board of directors}, an institution that has been refined in the private sector over many years.

According to a study in the Wall Street Journal, small boards tend to be the most effective at driving high shareholder returns\cite{Lublin}. Here we create a board consisting of 6 councilors along with the president who act as a team. They may differ in their style and philosophical outlook, but they have the same constituency and mission. This is similar to how a board of directors has a fiduciary duty to act in the interests of the common shareholder. The Council therefore does not make a good replacement for institutions such as the House and Senate whose members represent adversarial interests. Instead, it should be tasked with ensuring that the government's assets are well cared for over the long run.

\begin{quoting}
\textbf{Section 3}

The purpose of the Executive Council is to protect the long term interests of the United States and the liberty of its people.

The Executive Council shall consist of 6 councilors and the president of the United States, all with an equal vote on the council. The president votes last on the council, and only if such a vote would affect the outcome of the resolution.
\end{quoting}

This is so that the president is not required to attend most Council meetings, and can be presented with resolutions after the fact.

\begin{quoting}
Each counselor shall serve for a term of 7 years, with staggered terms so that one term ends and another simultaneously begins once every 14 months. Eligibility for councilor shall be the same as eligibility for senator.
\end{quoting}

In creating this council, we can take lessons from similar governing arrangements\cite{Khanna}. The Executive Council of New Hampshire, for example, has been particularly effective as the state's \quotes{board of directors}\cite{Hahn-Burkett}. Although it is made up of Democrats and Republicans, the debates are less partisan\cite{Timmins}. More research is needed to determine a causal link, but New Hampshire has maintained high levels of economic freedom and fiscal discipline in relation to similar states\cite{Ruger}.

Switzerland is also run by a federal council, with the Swiss being generally satisfied with their form of government and democratic institutions according to the Organisation for Economic Co-operation and Development\cite{Kaufman}.

There are also cautionary tales in council design. The defunct New York City Board of Estimate was made up of officials from other elected positions and its powers included permissions for permits and land use that encouraged trading political favors rather than expertise\cite{Purnick}. It was ultimately ruled unconstitutional for its malapportionment\cite{Board_of_Estimate}.

\subsection{The Proposals and Prophecies of George Mason}

George Mason was a founding father at the constitutional convention. He was a strong advocate of the Bill of Rights, having previously authored one of its predecessors in the Virginia Declaration of Rights. He even refused to endorse the constitution until the Bill of Rights was included.

Mason came to the convention wary of allowing a unitary executive to preside over the United States. He reasoned that, \quotes{If strong and extensive Powers are vested in the Executive, and that Executive consists only of one Person, the Government will of course degenerate... into a Monarchy… These Sir, are some of my Motives for preferring an Executive consisting of three Persons rather than of one.}\cite{Mason}

The convention rejected Mason's proposal. Perhaps they observed that effective organizations tend rely on a unitary executive. Undeterred, Mason tried again as the convention came to a close. He implored the delegates to at least accept the best of both worlds by providing the president with a council.

He wrote: \quotes{The President of the United States has no Constitutional Council, a thing unknown in any safe and regular government. He will therefore be unsupported by proper information and advice, and will generally be directed by minions and favorites; or he will become a tool to the Senate--or a Council of State will grow out of the principal officers of the great departments; the worst and most dangerous of all ingredients for such a Council in a free country; From this fatal defect has arisen the improper power of the Senate in the appointment of public officers, and the alarming dependence and connection between that branch of the legislature and the supreme Executive.}\cite{Mason_Objection}

Mason correctly points out the confusion that the framers had with respect to the Senate by making it a legislative body with executive characteristics. His warning of a \quotes{Council of State} seems particularly precinct today in light of the vast powers of the intelligence agencies and permanent administrative state.\cite{Cooper}

In Mason's final proposal to the convention, \quotes{...he suggested that a privy Council of six members to the president should be established; to be chosen for six years by the Senate, two out of the Eastern two out of the middle, and two out of the Southern quarters of the Union, \& to go out in rotation two every second year; the concurrence of the Senate to be required only in the appointment of Ambassadors, and in making treaties. which are more of a legislative nature.}\cite{Madison}

He was careful to note the differences between the Senate and the Council, offering advice on which should do what. He \quotes{considered the Senate as too unwieldy \& expensive for appointing officers, especially the smallest...}

Our proposal is similar to Mason's in that it envisions a council that turns over gradually. However, we note that regional representation is already achieved through the House and the Senate. Instead, the differentiating factor in council membership will be time. Elections suffer from a recency bias with their outcomes being disproportionately affected by events closer to the election date\footnote{There is a reason for the phrase \quotes{October Surprise}.}. By staggering its terms, the council can counterbalance the recency bias and short-lived political fads\footnote{Every well engineered information system must account for anomalies and trends, something that I am familiar with through designing recommendation systems.\cite{Yang}} that come from the electoral process. We also increase the 6 year term to 7 and rotate every 14 months instead of every 12 to cycle through different moments in the natural and political seasons.

\subsection{Electing Councilors}

We now present a unique method for electing councilors, which is carefully designed to give the Council the properties it needs to serve its purpose well.

\begin{quoting}
The election of a new councilor shall occur in several steps, with the precise schedule set by appropriate legislation.

Each state shall appoint an electoral committee for the next term in a manner according to state law. Each electoral committee shall compile a list of candidates in order of preference and in accordance with state laws. These committees shall transmit their respective lists as ballots to the president of the Senate.  The president of the Senate shall, in the presence of the Senate, open all the certificates and read them publicly. The votes of the committees shall then be tabulated.
\end{quoting}

States might use popular elections to inform and constrain the committees. If the Senate is ever reapportioned, a state's ballot could be counted multiple times.

\begin{quoting}
A ballot is considered to prefer one candidate over another when the former candidate appears higher on the list than the latter candidate, or if the former candidate appears on the list while the latter candidate is absent.

One candidate is considered preferred to another candidate as a whole if the number of ballots that prefer the former to the latter is greater than the number of ballots that prefer the latter to the former.

Should there exist a candidate that is preferred to every other candidate as a whole, this individual is considered a clear winner and becomes the next councilor.
\end{quoting}

The electoral system here is looking for a Condorcet winner, named after an 18th century French mathematician and politician who did foundational work in social choice theory. A clear  winner may not exist, but if it does this individual would be the best choice for the group.

The reason to use a Condorcet method here is that it encourages and ultimately selects strong independent candidates in a partisan environment. For example, suppose that 40\% of the committees prefer the partisan Democrat, 40\% prefer the partisan Republican, and 20\% support a strong independent candidate. Under instant runoff voting, that independent candidate would be eliminated in the first round. Under the Condorcet method, the independent could win because we can imagine both the Democrats and the Republicans preferring them to the other side.

The tabulation in this system is a bit more complex than alternative ranked-choice methods like instant runoff voting. It also requires the voters to have researched and ordered all the candidates. While this system may not be workable on a large-scale, it becomes possible through the electoral committees.

\begin{quoting}
Should no such candidate exist, the tabulators shall then determine the smallest group of candidates who are preferred to every candidate outside the group. The Senate shall then decide among these candidates unless there is a law that provides a formula for choosing among these candidates from the ballot rankings.
\end{quoting}

If there is no winner, this means that there is either a tie or a cycle of preferences. If this is the case, we then look for the \quotes{Smith set}\cite{Smith_Set}, which is the set of potential winners. There are many existing ways to choose among the Smith set, with names like the Schulze method\cite{Schulze} and minimax. We can allow this method to be changed through legislation, but it can only determine the outcome based on information from the committee lists and not on other characteristics such as \quotes{who is older} or \quotes{who won a coin toss}.

By default, we have a contingent election in the Senate to ensure a final decision. With these elections occurring frequently, data will begin to accumulate to inform election law.

\begin{quoting}
After the first councilor is election, a method for filling the five vacant partial terms may be set by law. The Council shall fill any other vacancy by setting a date for the electoral committees from that term to present new lists to the Senate.
\end{quoting}

Why have the states appointing electoral committees instead of submitting the ranking directly? It is because these committees will be consulted on any vacancies that occur. It is a way to respect the 7 year term and disincentivize politically-motivated removals from office, as the same electoral committees will choose the replacement. It also allows a councilor to resign while minimizing the likelihood of some vast partisan swing.

\subsection{Powers of the Executive Council}

\begin{quoting}
With a majority, the Council may recommend the repeal of legislation signed into law no less than six years prior. With such a recommendation, both houses of congress have 90 days to confirm the legislation with a majority vote or the repeal is effective.
\end{quoting}

Legislatures are have not been good at repealing outdated laws that adds cruft to the statute books. The Council helps here. The next clause echoes article 56 of the New Hampshire constitution\cite{New Hampshire Constitution} for ensuring a sound fiscal policy.

\begin{quoting}
No money shall be issued out of the treasury (except as may be appropriated for the redemption of bills of credit, or treasurer's notes, or for the payment of interest) but by the president with advice and consent from the Council that this disbursement is consistent with Article 1 Section 9 pertaining to treasury disbursements.

The Executive Council shall have the power to nominate justices of the Supreme Court, as well as ambassadors and other public ministers, subject to the advice and consent of the Senate.

The Executive Council shall inherit from the previously constituted senate the role of providing advice and consent to presidential appointments.
\end{quoting}

The critical impact that Supreme court appointments have on constitutional law suggests that it ought to be debated by an adversarial body like the Senate. However, the Council is a more fitting institution to make these nominations due to its independent nature. For other appointments, the Senate already gives the president much leeway. The Executive Council would be better positioned for the \quotes{advice} part of this task, with their long term outlook and national constituency.

\begin{quoting}
The Council attains a supermajority on a resolution if at least three fifths of the members vote in the affirmative, or equivalently 5 members from a full council of 7.
\end{quoting}

The following actions are very serious and therefore require the legitimacy of a supermajority. This provides the Council with ways to counter profligate spending and unconstitutional legislation.

\begin{quoting}
With a supermajority the Council may reduce the amount of any appropriation made by law, so long as the disbursements from that appropriation are under their power of review by law\footnote{They cannot block interest payments and other government promises.}.

With a supermajority, the Council may declare any legislation unconstitutional and shall include their reasoning in any such declaration. The legislation is immediately null and void. However, any member of congress has standing to challenge this reasoning at the Supreme Court which has the authority to evaluate and overrule the decision.
\end{quoting}

\subsection{Reining in the Administrative State}

There is growing recognition that the unelected administrative state is an affront to the values of individual liberty and due process that the American government was supposed to uphold. The permanent administration comes in the form of agencies such as the SEC, the EPA, the FDA, the FCC, the FTC, the IRS, the CDC, and OSHA. This is but a sampling of agencies embroiled in some recent controversy. Books and volumes could be assembled documenting the abuse of power and appearance thereof within these organizations. This is not to say that their officers or employees are all somehow corrupt or unaccountable, but that their accountability is not properly weighted towards the people that they are supposed to serve or even the people's elected representatives.

In a sense, congress has outsourced some of its legislative power to these agencies by allowing them to make specialized rules. As executive agencies they additionally have the power to choose who to target for enforcement and when to look the other way. The political appointees of these agencies - often with the \quotes{Commissioner} title - have split accountability being nominated by the president and approved by the Senate. Sometimes these commissioners serve terms lasting many years! And under these political appointees are policy-making officials who are part of the permanent bureaucracy.

Similarly situated are the intelligence agencies: the CIA, FBI, and NSA. Not only do these organizations suffer from the same issues, but they appear to use their positions to unduly influence elections and elected officials. According to Senate majority leader Chuck Schumer\cite{Stanley}, \quotes{you take on the intelligence community, they have six ways from Sunday at getting back at you.} Equally alarming is that these agencies systematically censor speech through intermediate platforms by claiming concern over misinformation and foreign propaganda\cite{Beanz}.

Recent efforts to rein this in have come from the judiciary\cite{Crowell} and the White House\cite{ScheduleF}. The judicial process takes years to resolve while presidential efforts have been stalled since the 2020 election.

How did we get here? The structure of our government allowed this to happen. The new Executive Council is in a better position than any existing institution to check the power of the unelected administrative state and intelligence agencies.

The president can currently remove some appointments but would have more difficulty changing policy directly\cite{Fairlie}. Perhaps more powers could be given to the president, but it would be unlikely for a single individual to wield that power in a thoughtful way that confers legitimacy. Therefore, we give the president the ability to overturn administrative policy with the consent of the Council.

\begin{quoting}
With either a supermajority or a simple majority that includes the president, the Council may alter or nullify any administrative policy originating in the executive branch. They may also remove administrative officers and employees who make policy. This does not remove any powers previously held by the president to act unilaterally in cases of policy and removal.
\end{quoting}

The supermajority rule is designed to give the council a moderating pushback on presidential actions. The bar is set high to prevent the appearance of a continual coup over the president from the Council.

\subsection{Impeachment and Compensation}

\begin{quoting}
Executive councilors are subject to the federal impeachment process. They shall be well compensated as set out by law which may include financial promises or derivatives thereof to reward fiscal and economic success as defined by congress.
\end{quoting}

This executive compensation package might seem strange at first, but it will ensure a strong pool of talent from many sectors of public and private life for any council election. This talent pool will lead to better fiscal management. Furthermore, the bonus structure (playing a similar role to stock options) could be used to provide further incentives.

One might argue that decisions by the Executive Council would be just as bad as the permanent bureaucracy that we have today. This is unlikely because of the incentive structure of the Council, their position within the government, and their manner of election. In addition to their reward structure, councilors are not beholden to any one department or party. The Council would attract a variety of professions and expertise in its membership. Any deficiency or blind spot would become a point of discussion for the next council election. As a focused group of long-term strategic thinkers, they are more likely to take historical lessons seriously. Ultimately, the Council's success should not be measured against some utopia, but in comparison to the current administrative regime.

\section{Electing the President}

We start by removing the anachronistic \quotes{natural born citizen} clause which the founders included to prevent European royalty from capturing the government. The phrasing comes from Orin Hatch\cite{Somin_Hatch}, with a reduction of the 20 year requirement as convincingly argued by Ilya Somin\cite{Somin_NBC}.

\begin{quoting}
\textbf{Section 4}

A person who has been a citizen of the United States for 7 years, is not ineligible to hold office as President of the United States by reason of not being a native-born citizen.
\end{quoting}

The Electoral College is currently proportioned by adding the number of Senate and House seats from each state. Now that we have an Executive Council whose electors are apportioned like the Senate, we can drop that term from the Electoral College to achieve a similar balance.

\begin{quoting}
The number of electors for president shall be equal to the number of representatives to which the state and territory may be entitled, and shall be elected in the same manner as its representatives. If the Electoral College fails to yield a majority vote for any candidate and a contingent election occurs in the House of Representatives, that contingent election shall apportion one vote per representative.
\end{quoting}

The pattern is now complete. The president will be chosen on the same basis as the House of Representatives, and the Council will be chosen on the same basis as the Senate. Both DC and the US territories will have representation in the Electoral College. This supersedes the 23rd Amendment.

The Electoral College will now represent regions uniformly and based on population with strong protections against partisan gerrymandering. However, this does not create a national popular vote. Organizing voters into districts and constituencies is still valuable in its fault tolerance. In other words, it limits the effect of local anomalies from the natural (weather, disasters) to the artificial (ballot harvesting, fraud, or intimidation).

\section{Summary}

There is a connection here with the recent interest in neural network architecture. These networks are organized into layers that are each filtered through an activation function which truncates one or both sides of its input. This architecture has proven successful in training intelligent software. We can therefore consider our Electoral College not a shameful relic of the past, but as the neural network of democracy. This forms a principled compromise by recognizing both the importance of this modularity of districts and concerns over apportionment.

In table~\ref{table:institutions}, we summarize the four institutions laid out in this document and their various properties. Of course, these ideas will be revisited many times in the future and feedback from across the political spectrum is needed in order to achieve the intended balance. This is why this proposal encourages a vigorous debate over election law in the courtrooms and the marketplace of ideas.

\begin{table}[ht]
\centering
\renewcommand{\arraystretch}{1.5}
\begin{tabular}{|c|c|c|c|c|}
\hline
\textbf{Institution} & \makecell{House of \\ Representatives} & Senate & Presidency & \makecell{Executive \\ Council}  \\
\hline
\textbf{Purpose} & Voice of the People & Voice of the States & Leadership & Foresight \\
\hline
\textbf{Branch} & Legislative & Legislative & Executive & Executive \\
\hline
\textbf{Term Length} & 2 Years & State Law & 4 Years & 7 Years \\
\hline
\textbf{Process} & Adversarial & Adversarial & Unitary & Cooperative \\
\hline
\textbf{Apportionment} & Population & States & Population & States  \\
\hline
\textbf{Electors} & Direct per District & State Law  & Electoral College & \makecell{Electoral\\Committees}  \\
\hline
\textbf{Winning Condition} & Plurality & State Law & Majority & Condorcet  \\
\hline
% you can continue adding rows in the same format
\end{tabular}
\caption{A summary of the proposed legislative and executive institutions and their properties. }
\label{table:institutions}
\end{table}

\section*{Acknowledgements}

I was fortunate to have Aaron Bell, my longtime friend and collaborator on The Local Maximum podcast, helped me hash out the provisions of this proposed amendment and fixed its initial shortcomings. Our conversations can be heard in episodes 283 through 285 of The Local Maximum, as well as various discussions on voting systems, social choice theory, and constitutions that we've had over the years. Jordan Marks, my cousin and Tax Assessor for San Diego County, provided his commentary and to encouraged me to think carefully and systematically about the potentials and pitfalls to having various boards and commissions in government.

\appendix
\section{Supplementing the House with Representative Assemblies}
\label{appendix:house}

The following is proposal establishes the creation of representative assemblies around the country which can be brought about through legislation or amendment. Assembly members are not required to familiarize themselves with legislation. They instead exist to hold their own representatives in the House accountable as citizens and to occasionally decide on referendums. While this idea seems to put into practice the democratic nature of our government that was promised, there are many open questions about its feasibility and necessity. I hope it will provoke discussion just like the main proposal.

\begin{quoting}
Each house district shall have a representative assembly with terms coinciding with the term for the House of Representatives. The size of each representative assembly shall be proportional to the population of that district. Elections for each assembly shall occur at-large with seats awarded proportionally to the preferences indicated by the voters.

Assemblies have the power to elect and recall their representative, to arrange sessions for their representative to address the assembly and answer questions, to elect a new representative in case of a vacancy, and to participate in assembly referendums.
\end{quoting}

This is not supposed to be an \quotes{indirect election} to the house. Voters will still know which house candidate they are voting for, with the added benefit of their preference being represented in their local assembly. Note the two types of proportionality here. The first ensures that the number of assembly members per capita is about even throughout the country. The second calls for a specific type of electoral system that can more easily represent minority voices. This proportional system could be brought about by voters selecting a party list, or more likely seats could be allocated to candidates based on their vote totals. The system would be an improvement over the current plurality voting system by giving a voice to third parties and independents in the assemblies while maintaining the moderation induced by single-district constituencies in the House itself.

\begin{quoting}
Members of representative assemblies shall be volunteers, only receiving reimbursement funds related to assembly business. Assemblies shall keep a light schedule and make reasonable accommodations for members in various life circumstances. Representative assemblies shall meet in public in their home district except when due to unusual conditions.

Assembly directives are rules governing the House of Representatives and the assemblies themselves. Assembly directives are voted upon in an assembly referendum. During a referendum, all assemblies including those for non-voting members meet to debate one or more resolutions and vote on them. The vote totals are transmitted to the House to tabulate the official total and determine if a majority of assembly members nationally have voted in favor.
\end{quoting}

Because the assemblies are very evenly apportioned by population and the proportional nature of their election cannot easily be manipulated through gerrymander, this referendum is truly representative in a way the House can only approximate. The idea here is that the assemblies can govern the House, particularly on questions of procedure and ethics, to ensure that it represents the people as best as possible. Note that Article 1 Section 5 states that the House should determine its own proceedings, so this change may require a constitutional amendment.

\begin{quoting}
The internal governing rules for proceedings in the House and the assemblies shall be decided in the following in order of precedence: the constitution, assembly directives, and each body’s own rules. Additionally, the house may make rules for the assemblies that override each assembly’s own rules but do not override assembly directives.
\end{quoting}

This document does not and should not decide on all the rules for governance of these assemblies. The initial rules would come from the house itself, which could later be codified into directives. These initial rules might include: providing a formula for the size of the assemblies, the proportional voting system for assemblies, how assemblies choose their representative, how to break deadlocks, requiring a supermajority for an assembly to remove their representative, and how to trigger a referendum.

\begin{thebibliography}{20}
\bibitem{Tucker}Tucker, Jeffrey. \href{https://www.theepochtimes.com/repeal-the-17th-amendment-now\_4909126.html}{\quotes{Repeal the 17th Amendment Now.}} The Epoch Times, December 7, 2022.

\bibitem{Virginia}McInerney, Virginia. \href{https://lonang.com/commentaries/conlaw/federalism/repeal-seventeenth-amendment/}{\quotes{Why the 17th Amendment Needs to Be Repealed.}} LONANG Institute, March 27, 2020.

\bibitem{Smith}Smith, Terry. \href{https://www.latimes.com/archives/la-xpm-2010-oct-22-la-oew-smith-17th-amendment-20101022-story.html}{\quotes{Why We Have, and Should Keep, the 17th Amendment.}} Los Angeles Times, October 22, 2010.

\bibitem{Komoroske}Komoroske, Alex. \href{https://medium.com/@komorama/on-schelling-points-in-organizations-e90647cdd81b}{\quotes{On Schelling Points in Organizations.}} Medium, December 31, 2021.

\bibitem{Allen_Milligan}Allen v. Milligan, 599 U.S. (2023)

\bibitem{Brewer}Brewer, Simon. \href{https://law.yale.edu/mfia/case-disclosed/back-basics-why-partisan-gerrymandering-violates-first-amendment}{\quotes{Back to Basics: Why Partisan Gerrymandering Violates the First Amendment.}} Yale Law School, Media Freedom \& Information Access Clinic, March 12, 2019.

\bibitem{Allen}Allen, Danielle. \quotes{Opinion | \href{https://www.washingtonpost.com/opinions/2023/03/28/danielle-allen-democracy-reform-house-representatives-districts/}{Just How Big Should the House Be? Let’s Do the Math}.} The Washington Post, May 23, 2023. 

\bibitem{30000} \quotes{\href{https://thirty-thousand.org/the-house-of-representatives-is-scalable/}{Learn Why a Larger House Will Be Much More Productive and Enable the Representatives to Better Serve Their Constituents}.} thirty-thousand.org, June 6, 2022.

\bibitem{Federalist10}Madison, James. \quotes{The Federalist Papers, No. 10.} November 22, 1787 (1787).

\bibitem{Federalist62}Madison, James. \quotes{The Federalist Papers, No. 62.} Independent Journal, February 27 (1788).

\bibitem{Eisinger}Eisinger, Vince. \quotes{Auxiliary Protections: Why the Founders' Bicameral Congress Depended on Senators Elected by State Legislatures.} Touro L. Rev. 31 (2014): 231.

\bibitem{Schiller}Schiller, Wendy J., and Charles Stewart III. \href{https://www.brookings.edu/wp-content/uploads/2016/06/Schiller_17th-Amendment_v7.pdf}{\quotes{The 100th Anniversary of the 17th Amendment: A Promise Unfulfilled?}}. Issues in Governance Studies at the Brookings Institute. May, 2013

\bibitem{Madison}Madison, James. Original Notes on Debates in the Congress of Confederation, 1787.

\bibitem{Senate}\href{https://www.senate.gov/about/origins-foundations/senate-and-constitution.htm}{\quotes{About the Senate and the Constitution.}} U.S. Senate: About the Senate and the Constitution, September 1, 2022.

\bibitem{Amar}Amar, Akhil Reed. \quotes{Philadelphia Revisited: Amending the Constitution Outside Article V.} U. Chi. L. Rev. 55 (1988): 1043.

\bibitem{Shapley}Shapley, Lloyd S., and Martin Shubik. \quotes{A method for evaluating the distribution of power in a committee system.} American political science review 48, no. 3 (1954): 787-792.

\bibitem{Gross}Gross, Benjamin R. \quotes{Solving the Electoral College.} PhD diss., Colorado College., 2010.

\bibitem{Lublin}Lublin, Joann S. \href{https://www.wsj.com/articles/smaller-boards-get-bigger-returns-1409078628}{\quotes{Smaller Boards Get Bigger Returns.}} The Wall Street Journal, August 29, 2014. 

\bibitem{Khanna}Khanna, Parag. \href{https://qz.com/876260/seven-presidents-are-better-than-one-why-the-oval-office-needs-a-round-table}{\quotes{Seven Presidents Are Better than One: Why the Oval Office Needs a Round Table.}} Quartz, January 10, 2017.

\bibitem{Hahn-Burkett}Hahn-Burkett, Tracy. \href{https://www.concordmonitor.com/What-is-the-Executive-Council-34817477}{\quotes{3-Minute Civics: What Is the Executive Council?}} Concord Monitor, June 23, 2020.

\bibitem{Timmins}Timmins, Annmarie. \href{https://www.nhpr.org/nh-news/2021-10-25/executive-council}{\quotes{The Behind-the-Scenes Power of N.H.’s Executive Council Is Now at Center Stage.}} New Hampshire Public Radio, October 25, 2021. 

\bibitem{Ruger}Ruger, William, and Jason Sorens. Freedom in the 50 states, 2013 edition: An index of personal and economic freedom. The Mercatus Center at George Mason University, 2013.

\bibitem{Kaufman}Kaufmann, Bruno. \quotes{\href{https://www.swissinfo.ch/eng/business/the-strengths-of-a--weak--swiss-government/48483858. }{The Strengths of a ‘weak’ Swiss Government}.} SWI swissinfo.ch, May 5, 2023. 

\bibitem{Purnick}Purnick, Joyce. \quotes{\href{https://www.nytimes.com/1986/02/16/weekinreview/the-board-of-estimate-and-its-critics.html}{The Board of Estimate and Its Critics}.} The New York Times, February 16, 1986.

\bibitem{Board_of_Estimate}Board of Estimate of NYC v. Morris, 489 U.S. 688 (1989)

\bibitem{Mason}Rutland, Robert. \quotes{The Papers of George Mason}, 3 vols. The University of North Carolina Press. Chapel Hill, NC. 1970.

\bibitem{Mason_Objection}Mason, George. \quotes{Objections to the Constitution of Government Formed by the Convention.} Allen and Lloyd, The Essential Antifederalist (1787): 12.

\bibitem{Cooper}Cooper, Charles J. \href{https://www.nationalaffairs.com/publications/detail/confronting-the-administrative-state}{\quotes{Confronting the Administrative State.}} National Affairs, Fall 2015.

\bibitem{Yang}Yang, Stephanie, and Max Sklar. \quotes{Detecting Trending Venues Using Foursquare's Data.} In RecSys Posters. 2016.

\bibitem{Smith_Set}Smith, John H. \quotes{Aggregation of preferences with variable electorate.} Econometrica: Journal of the Econometric Society (1973): 1027-1041.

\bibitem{Schulze}Schulze, Markus. \quotes{The Schulze method of voting.} arXiv preprint arXiv:1804.02973 (2018).

\bibitem{New Hampshire Constitution}Constitution of New Hampshire, Part 2, Article 56

\bibitem{Stanley}Stanley, Jay. \href{https://www.aclu.org/news/national-security/do-us-politicians-need-fear-our-intelligence}{\quotes{Do U.S. Politicians Need to Fear Our Intelligence Agencies?: ACLU.}} American Civil Liberties Union, February 27, 2023.

\bibitem{Beanz}Beanz, Tracy. \href{https://www.uncoverdc.com/2023/05/24/the-lead-up-to-the-hearing-missouri-v-biden/}{\quotes{The Lead up to the Hearing: Missouri v. Biden.}} UncoverDC, May 25, 2023. 

\bibitem{Crowell}\quotes{\href{https://www.crowell.com/a/web/bomv5ATK9LZPNrBA51skWq/4TtiyY/Regulatory-Forecast-2020-Administrative-Law-Crowell-Moring.pdf }{Administrative Law – The Supreme Court and the President Rein in the `Administrative State`}.} Regulatory Forecast 2020, 2020. 

\bibitem{ScheduleF}Executive Order No. 13957. (2020) \href{https://www.
federalregister.gov/documents/2020/10/26/2020-23780/creating-schedule-f-in-the-excepted-service}{Creating schedule F in the excepted service}, October 21, 2020.

\bibitem{Fairlie}Fairlie, John A. \quotes{The Administrative Powers of the President.} Michigan Law Review (1903): 190-210.

\bibitem{Somin_Hatch}Somin, Ilya. \href{https://reason.com/volokh/2020/08/16/orrin-hatchs-constitutional-amendment-to-abolish-the-natural-born-citizen-clause/}{\quotes{Orrin Hatch’s Constitutional Amendment to Abolish the Natural Born Citizen Clause.}} Reason.com, August 29, 2020.

\bibitem{Somin_NBC}Somin, Ilya. \href{https://reason.com/volokh/2020/08/14/why-we-should-abolish-the-requirement-that-the-president-must-be-a-natural-born-citizen/}{\quotes{Why We Should Abolish the Requirement That the President Must Be a ‘Natural Born Citizen’ [Updated].}} Reason.com, August 29, 2020. 

\end{thebibliography}

This document along with revisions is posted at github as https://github.com/maxsklar/great-compromise-ammendment. See readme for contact information. No funds, grants, or other support was received.
\end{document}
