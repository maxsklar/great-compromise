\documentclass{article}
\usepackage{amssymb}
\usepackage{amsmath}
\usepackage{algorithm}
\usepackage{algpseudocode}
\usepackage{pgfplots}
\usepackage{multicol}
\usepackage[hmarginratio=1:1,top=32mm,columnsep=20pt]{geometry}
\usepackage{fullpage}
\usepackage{pdflscape}
\usepackage[toc,page]{appendix}
\usepackage[shortlabels]{enumitem}
\usepackage[font=itshape]{quoting}
\usepackage{tabularx} % for 'tabularx' environment and 'X' column type
\usepackage{makecell}
\usepackage{hyperref}

\begin{document}
\parindent=0in
\parskip=12pt

\title{
  Toward a New Great Compromise\\
  \large{
    A Constitutional Plan to Restore Federalism and Democracy
  }
}

\author{Max Sklar\\ Local Maximum Labs \\ July, 2023}
\date{}

\maketitle
\thispagestyle{empty}

There has not been a structural amendment to the US constitution for many decades. A convention of the states could make this happen, but such a convention has never occurred. However, as the contradictions in our political structure become more apparent, there is an increasing possibility that this could happen. Therefore, why not continue to discuss and debate how our government is structured on the most basic level?

Our approach should neither deify the founding fathers nor dismiss them as irrelevant relics. Instead, we should seek to understand why they made the decisions they made, how those decisions have echoed throughout our history, and what that suggests about changes we should make today.

\section{The Emerging Crisis in Form of Government}

Americans have much to demand of their form of government. It should be a federal republic with checks and balances to maintain the rule of law. It should be democratic both in the derivation of its power through elections and in its approach to transparency and open debate. This includes the ability for multiple parties to compete, respect for civil liberties, and the all-important freedom of speech.

Many Americans see their liberties are under attack and liberal democracy slipping away, but they cannot agree on where the problem lies. Part of the fracture concerns centralized urban populations versus widely distributed rural populations. The rural faction would be better served by a return to federalism, which would be signified by a repeal of the 17th amendment\cite{Tucker}\cite{Virginia}. The 17th amendment will undoubted find support among the progressive faction which has benefited from urban votes and sees direct democracy as a positive good\cite{Smith}.

Neither side is going away, and neither has enough support to permanently impose their will on the other. In the spirit of the “Great Compromise” of 1787 that founded this country, we should seek to create a plan to that has properties attractive to both the left and the right, and that could garner broad support after public debate. This work provides a rough attempt at this, and discusses the questions that the proposal would raise. Every conversation for great improvement must start from an underdeveloped state. I seek your feedback for improvements.

\section{A New Great Compromise}

The rest of this document will draft a constitutional amendment and explain the reasons for these decisions. The proposed changes are as follows.
\begin{enumerate}
  \item The \textbf{house of representatives} will include representation for DC and the territories, and a stronger judicial oversight to evaluate election law and district maps. The goal will be to represent the diverse interests of the people.
  \item Members of the \textbf{Senate} are chosen by their respective states and are in office at-will with states being able to recall them at any time for any reason, thereby establishing the accountability they currently lack.
  \item A new \textbf{Executive Council} will be created to replace the aspects of the senate that require continuity and long terms. Selected by a senate-like electoral college, this council will provide some professional governance in the executive branch. The first national Condorcet election will be introduced that will encourage independent candidates.
  \item The electoral college for the \textbf{president} will be apportioned by population like the house of representatives.
 \item Each institution is given a clearer purpose and mission.
\end{enumerate}

\section{Institutional Purpose and the Founder’s Confusion}

Institutions that have a clear purpose and mission are more likely to be effective. It becomes more difficult for special interests and ideologues to undermine them over time.

As we will discuss in part~\label{section:Senate}, the delegates at the constitutional convention had competing ideas for the senate and differed on whether it was going to be the “voice of the states” or the “more professional upper chamber”. They ended up combining both, preventing the senate from serving either purpose as effectively as it could. This confusion has echoed throughout history, eventually leading to the 17th amendment.

By separating out these concerns, we can have a government with democracy, federalism, and continuity. Here is how it could break down:
\begin{enumerate}[I]
\item The purpose of the house of representatives is to represent in its members the broad spectrum of interests and ideas of the people of the United States in the legislative branch.
\item The purpose of the senate is to represent the interests of the individual states in the legislative branch.
\item The purpose of the executive council is to be a professional board to assist the president and provide for the long term interests of the United States.
\end{enumerate}

\section{House of Representatives}

We now begin the proposed amendment, indented and italicized.

\begin{quoting}
\textbf{Section 1}

The purpose of the house of representatives is to represent in its members the broad spectrum of interests and ideas of the people of the United States in the legislative branch.
\end{quoting}

With that established, it makes sense to ensure that all citizens get a voice in the house of representatives, including those who do not reside in any state.

\begin{quoting}
In addition to apportioning seats in the house of representatives to each state based on population, seats shall be apportioned for the District of Columbia and United States territories. The seats allocated to each entity shall be based on its population just the same as if it were a state. Territories with populations less than half of the number of house seats per capita shall instead receive a non-voting member.
\end{quoting}

This provides representation to the population in Puerto Rico and DC. This would mean approximately 1 seat for DC and 3 seats for Puerto Rico, with the smaller territories receiving non-voting members.

We next turn to how the states and territories elect their representatives. Elected representatives have a clear conflict of interest because any reform will come with advantages and disadvantages for their own electoral success. The supreme court has therefore taken a strong role in election law, but they continue to grapple with defining that role.

The recent decision in Allen vs. Milligan is an example of where the court can intervene in unfair electoral maps. It is certainly not the first. However, this intervention has been fairly limited\cite{Brewer}. The courts have legal basis to protect against racial gerrymandering but they cannot do as much with regards to electoral systems that merely convey an unfair partisan advantage. We seek to expand the role of the courts here.

\begin{quoting}
The manner in which states elect their representatives shall be designed to produce a delegation that is broadly and proportionally representative of the population of that state and its various communities within the constraint of the delegation size. This manner of election falls first to the states, and then to federal law which may make or alter such regulations.

Any elector or group of electors shall have standing to bring suit in the federal judiciary with claims that these election regulations are misaligned with the purpose of such regulations. In considering such cases, the judiciary shall take into account the totality of
\begin{itemize}
\item the historical precedent
\item the changing facts, particularly concerning the electorate, which may degrade a once-valid electoral system
\item our evolving understanding of the empirical properties of electoral law as discovered through observation of elections in practice.
\item our evolving understanding of the theoretical properties of electoral law as discovered through logical deduction
\end{itemize}
\end{quoting}

This would allow judges to take a more active stand on election law. Over time, the courts will have a freer hand in correcting problems with redistricting and partisan gerrymandering. They will use past decisions as a starting point, but will not be wholly constrained by them.

Electoral systems are a hard problem, and no magic solution exists to achieve mathematical perfection. Some might say an abolition of districts altogether in favor of a proportional system would eliminate these problems. They may be right, but there are certainly pitfalls to such a system that are at least arguable.

We can instead set ourselves up to evolve with a north star statement like: “broadly and proportionally representative of the population of that state and its various communities, as well as can be done given the delegation size”. Courts will grapple with the idea of “various communities” as each group presents their case for representation, and will come to a set of principles for evaluating these claims over time.

Congress should consider increasing the house of representatives in order to provide representation for our large population. Several plans have been introduced in order to do this, from a modest increase to a gargantuan one\cite{Allen}. A plan provided by thiry-thousand.org even calls for an increase in the house in order to satisfy the low population-to-representative ratio envisioned at constitutional convention\cite{30000}.

Such plans must be debated and analyzed. Madison urges caution in Federalist 10 when he writes, "the representatives must be raised to a certain number, in order to guard against the cabals of a few; and that, however large it may be, they must be limited to a certain number, in order to guard against the confusion of a multitude."\cite{Federalist10}

\section{Senate}
\label{section:Senate}

The original constitutional convention was split between the senate existing to “represent the states” and to be a “long-term, more professional, upper house”. Hamilton and Madison saw it as the latter, with Hamilton originally proposing life-long terms, and Madison originally proposing that they be appointed by the house. Even the delegates who proposed senators be elected by the state legislatures were not all doing it for reasons of preserving federalism. It was john Dickinson who proposed the election by state legislatures at the convention. According to the notes of Madison, Dickinson "wished the Senate to consist of the most distinguished characters, distinguished for their rank in life and their weight of property, and bearing as strong a likeness to the British House of Lords as possible; and he thought such characters more likely to be selected by the State Legislatures, than in any other mode." He additionally sell his proposal as a way to maintain some authority for the states.

In the end, Dickinson prevailed at the convention. He also sold his proposal as Although it was not his original plan, Madison still promoted state legislature appointments of senators as a net positive in Federalist 62 when he wrote, “It is recommended by the double advantage of favoring a select appointment, and of giving to the State governments such an agency in the formation of the federal government as must secure the authority of the former, and may form a convenient link between the two systems.”\cite{Federalist62}

So here we do have a justification in securing a degree of agency for the states, but it is never presented as the sole reason for state legistlature approval. Madison calls the link merely "convenient" rather than "crucial". It is no wonder that the 17th ammendment was passed so easily. The political class was never commited to the structure to begin with!

The 17th ammendment was passed by support from the progressive movement in 1913 with the hopes that it would lead to more democratic accountability to the federal government\cite{Eisinger}. Has it delivered on its promise? A report by the Brookings Institute on 100th anniversay of the 17th amendment found that none of this accountability could be verified by political data\cite{Schiller}. Senators elected directly seem to be no more responsive to voters than senators elected indirectly.

And why should they be? They have a six year term. They usually have millions of constituents with diluted voting power, and can get re-elected more easily by courting doners than by listening to the concerns of individuals. The progressive authors of the 17th amendment mistakenly believed that it was indirect election of the senators that made them unaccountable without considering the whole founding design.

The convention delegates considered term length as an imporant feature of the institutions they created. Roger Sherman, Connecticut delegate and author of the “great compromise“ warned against long term lengths. He said, “Government is instituted for those who live under it. It ought therefore to be so constituted as to not be dangerous to their liberties. The more permanency it has the worse if it be a bad government.”\cite{Madison}

Here we divide the senate into 2 parts: the senate will now represent the states, and the executive council will provide long-term continuity.

\begin{quoting}
\textbf{Section 2}

The purpose of the senate is to represent the interests of the individual states in the legislative branch.

Each state shall appoint one senator, and may appoint any number of alternate senators. Such alternates may engage in Senate business in the senator’s absence, according to a clear order of succession indicated by the constituent state.
\end{quoting}

This section starts out with a change to article 1 section 3 on the composition of the senate, and also renders inoperative the 17th amendment. Some senators today are there to provide straight party-line votes without any talent as a legislator. Others are political relics who no longer serve their state’s interests but who can coast from familiarity with voters and donors. This new senate would likely contain fewer of these types, and its members would have to be incredibly familiar with the effects of federal legislation on their particular state government.

The founders wanted multiple senators because otherwise there would be too many absences and states not represented. Nowadays, ease of communication and travel make absences less necessary during key votes. Still, senators will not always be available, so we will allow alternates. At the constitutional convention, Maryland delegate Martin Luther called for 1 vote per state, as it was in the continental congress and the congress under the articles of confederation\cite{Senate}.

\begin{quoting}
The method of selecting and removing senators shall be determined by the laws of the state from which they represent. A state may appoint, recall, or replace their senate delegation at any time in accordance with the laws of that state. The senate shall determine reasonable and orderly procedures for updating its roster and handling rotations as it receives notices of appointment, recall, and replacement from the states.

Absent relevant state laws, senators shall be selected by the state legislature. If the state legislature does not select a senator, the highest executive officer in the state can appoint a senator to serve until the legislature appoints one. This method of appointment is overruled only by state laws that are passed after this amendment is ratified.
\end{quoting}

The authors of the 17th amendment were concerned about deadlocks that occurred where no senator was chosen. This solves it.

We next change article 1 section 6 of the constitution concerning the salaries of senators.

\begin{quoting}
Senators and alternate senators shall not receive any compensation from the treasury of the United States. They may receive compensation from the state they represent, as prescribed by the laws of their constituent state.
\end{quoting}

Madison wanted senators to have longer terms in order to provide stability and continuity in government. Because of this, salaries needed to come from the federal government, so that a state could not effectively recall a senator (or exert undue influence) by halting payments. Now that the senator serves at the pleasure of their state, that senator can get paid by the state.

\begin{quoting}
The senate shall retain the power to provide advice and consent on justices of the supreme court, to try impeachments, and their full role in the legislative process.
\end{quoting}

\section{Executive Council}

Today we would be hard pressed to find support for a “House of Lords“ as John Dickinson and others wanted.

Instead, we can recognize the value in having elected positions where terms are long and removal is unlikely. It allows these officeholders to focus on the job at hand rather than reelection campaign, and to persue long term goals that could replecate the decision-making processes preferred by owners of the county's assets rather than renters. It also provides a sense of continutity and stability to allow for institutional knowledge to be retained.

The senate was originally tasked with this role with their 6 year terms. It's debatable whether the senate ever took on this role effectively. Under this proposal, the senate may have some members with long tenure, but they will be continually held accountable to the constituents in their state. So, to fulfill the role we are looking for, we create the executive council and place it in the executive branch. The model will be a board of directors, an institution that has been tested in the private sector for many years in order to properly protect after shareholder value.

According to a study in the Wall Street Journal, small boards tend to be the most effective at driving high returns\cite{Lublin}. We therefore create a board of 6 councilors and the president who act as a team. They may differ in their style and philosophical outlook, but they have the same constituency and mission. This is similar to how a board of directors has a fiduciary dury to act in the interests of the common shareholder.

This contrasts with the house and the senate whose members represent adversarial interests. The powers of the executive council must be debated thoroughly, because such a council does not make a good replacement for democratic institutions such as the house. Instead, it should be used as an assist to ensure that the countries assets and budget are well cared for over the long run.

\begin{quoting}
\textbf{Section 3}

The executive council’s purpose is to protect the long term interests and well-being of the Union.

The executive council shall consist of 6 councilors and the President of the United States, all with an equal vote on the council. The president may not vote until all of the other councilors have voted, and only if such a vote would affect the outcome of the resolution.
\end{quoting}

It does not require the President to preside over the council as the VP does the senate, and it can meet without the President.

\begin{quoting}
Each counselor shall serve for a term of 6 years, with staggered terms so that one term ends and another simultaneously begins once per year, with a more precise schedule set by law. Eligibility for councilor shall be the same as eligibility for senator.
\end{quoting}

In creating this council, we can take lessons from other councils and private sector boards\cite{Khanna}. The executive council of New Hampshire, for example, has been particularly effective as the “board of directors” for the New Hampshire state governor\cite{Hahn-Burkett}. Although it is made up of democrats and republicans, the debates are less partisan\cite{Timmins}. More research is needed to determine the causal link, but New Hampshire has maintained high levels of economic freedom and fiscal discipline in relation to similar states\cite{Ruger}.

The government of Switzerland is also run by a federal council, with the Swiss being generally satisfied with their form of government and democratic institutions according to the Organisation for Economic Co-operation and Development\cite{Kaufman}.

The defunct New York City Board of Estimate is a cautionary tale in council design. This board was made up of officials from other elected positions and its powers included permissions for permits and land use that encouraged trading political favors rather than expertise\cite{Purnick}. It was ultimately ruled unconstitutional for its malapportionment\cite{Board_of_Estimate}.

\subsection{George Mason's Proposals}

There were several council-like proposals at the constitution convention.

George Mason wrote disapprovingly of a unitary executive when he said, "If strong and extensive Powers are vested in the Executive, and that Executive consists only of one Person, the Government will of course degenerate, (for I will call it degeneracy) into a Monarchy:A Government so contrary to the Genius of the People, that they will reject even the Appearance of it. … These Sir, are some of my Motives for preferring an Executive consisting of three Persons rather than of one."\cite{Mason}

The convention decided against Mason's original proposal in favor of a unitary executive. It is likely that they observed that most governments and effective organizations rely on some kind of chief executive, even though that chief executive may range from an absolute dictator to a symbolic figurehead. As the convention wore on, Mason appears to accept the idea of a unitary executive, or at least be resigned to the fact. However, as the convention came to a close, he implored the delegates to consider that they can have the best of both worlds by providing the president with a council.

He writes: “The President of the United States has no Constitutional Council, a thing unknown in any safe and regular government. He will therefore be unsupported by proper information and advice, and will generally be directed by minions and favorites; or he will become a tool to the Senate--or a Council of State will grow out of the principal officers of the great departments; the worst and most dangerous of all ingredients for such a Council in a free country; From this fatal defect has arisen the improper power of the Senate in the appointment of public officers, and the alarming dependence and connection between that branch of the legislature and the supreme Executive.“\cite{Mason_Objection}

Here, Mason points out the confusion that the founders had with respect to the senate by making it a legislative body with executive characterists. His warning of a “Council of State“ seems particularly precient today in light of the vast powers of the intelligence agencies and permanent administrative state.

Mason's final proposal to the convention is documented in Madison's notes as “...he was averse to vest so dangerous a power in the President alone. As a method for avoiding both, he suggested that a privy Council of six members to the president should be established; to be chosen for six years by the Senate, two out of the Eastern two out of the middle, and two out of the Southern quarters of the Union, \& to go out in rotation two every second year; the concurrence of the Senate to be required only in the appointment of Ambassadors, and in making treaties. which are more of a legislative nature.“

He was careful to note the differences between the senate and the council, offering advice on which should do what. He “considered the Senate as too unwieldy \& expensive for appointing officers, especially the smallest...“

Like George Mason, our proposal here does making a council that turns over gradually. However, we note that representation by region is already achived through the house and the senate. Instead, the differentiating factor in council appointments will be time. Elections suffer from a recency bias, where their outcomes tend to be disproportionally affected by events closer to the election date than further in the past. There's a reason why the phrase “October Surprise“ or even “November Surprise“ are used in relation to the presidential election. By staggering council elections, the council as a whole can counterbalance the recency bias that comes from the electoral process.

\subsection{Electing Councilors}

We now turn to a unique method of electing councilors, which is designed to give the council the properties it needs to best serve its purpose.

\begin{quoting}
The council election shall occur in several steps, with the schedule for each step set by appropriate legislation.

In the initial nominating round, each state legislature shall submit to the senate a list of candidates for the executive council. The senate shall compile this list into an official nominee list in a manner set by law, with each state guaranteed to have their first choice of eligible nominee on that list.

In the next phase, each state shall appoint an electoral commitee for the next term in the same manner as it appoints its senators. Each electoral committee shall compile a single ordering of most preferred to least preferred candidate. Such ordering may be constrained by state laws as respecting the outcome of a poll or legislature directive. These committees shall transmit their respective lists to the president of the senate.  The President of the Senate shall, in the presence of the Senate and House of Representatives, open all the certificates and read them publicly. The votes of the commitees shall then be tabulated.

A ballot is considered to prefer one candidate over another when the former candidate appears higher on the list than the latter candidate, or if the former candidate appears on the list while the latter candidate is absent.

One candidate is considered preferred to another candidate by the electoral college as a whole if the number of ballots that prefer the former candidate to the latter is greater than the number of ballots that prefer the latter to the former.

Should there exist a candidate that is preferred to every other candidate by this electoral college, this individual is considered a clear winner and becomes the next councilor.
\end{quoting}

The electoral system here is looking for a Condorcet winner, named after an 18th century French mathematician and politician who did foundational work in social choice theory. This winner may not exist, but if it does, surely this would be the best choice for the group.

The reason why we prefer a Condorcet method here is that it encourages and ultimately selects strong independent candidates in a partisan environment. For example, suppose that 40\% of the commitees prefer the partisan democrat, 40\% prefer the partisan republican, and 20\% support a strong independent candidate. Under instant runoff voting, that independent candidate would be eliminated in the first round. Under the Condorcet method, the independent could win because we can imagine both the democrats and the republicans prefer them to the other side.

The tabulation in this system is a bit more complicated than alternative ranked-choice methods like instant runoff voting. It requires electoral commitees to have researched and ordered all the candidates. This may not be a workable system in a large-scale election, but through an electoral college it becomes possible. The orderings could come from the result of some state-wide election, or it could come from the state legislature or the commitees themselves. The tabulation becomes much easier when there are only 50 ballots to see if such a winner exists.

\begin{quoting}
Should no such candidate exist, the tabulators should then determine the smallest group of candidates who are preferred to every candidate outside the group. The senate shall then decide between these candidates unless there is existing law that provides a formula for choosing among these candidates from the ballot rankings.
\end{quoting}

If there is no winner, this means that there is either a tie or a cycle of preferences. If this is the case, we then look for something called the “Smith set”, which is the set of potential winners. There are many existing ways to choose among this set, with names like the Schulze method\cite{Schulze} and minimax, but it is still unclear which would lead to the best result in this case. Therefore, we can allow this decision to be changed through legislation. Note that the law can only determine the outcome based on what is on the ballot and not on other characteristics such as “who is older” or “who won a coin toss”.

By default, we have a contingency election in the senate. Through our experience with the presidential electoral college, we may guess that this won’t be used very often. But given the differences here, it may actually become more important. With one of these elections occurring every year, data will begin to accumulate that could inform election law.

\begin{quoting}
When there is a vacancy on the council, the council shall fill that vacancy by setting a date for the electoral commitee from that term to present new lists to the senate.
\end{quoting}

Why do we want the states to appoint electoral commitees and not just compile the ranking directly? It is because these electors will be consulted on any vacancies that occur. It’s a way to respect the 6 year term and disincentivize removals from office of council members for political reasons, as the same electoral commitees will choose the replacement. It also allows a councilor to resign while minimizing some vast partisan swing.

\subsection{Powers of the Executive Council}

\begin{quoting}
The executive council has the following powers with a majority vote:

To recommend a repeal of legislation signed into law no less than six years prior. With such a recommendation, both houses of congress have 90 days to confirm the legislation with a majority vote or the repeal is effective.
\end{quoting}

The next clause echoes a provision in the New Hampshire state constitution article 56\cite{New Hampshire Constitution} for ensuring a sound fiscal policy.

\begin{quoting}
No money shall be issued out of the treasury (except as may be appropriated for the redemption of bills of credit, or treasurer´s notes, or for the payment of interest) but by the president with advice and consent from the council that this disbursement is consistent with Article 1 Section 9 pertaining to treasury disbursements.

The executive council shall inherit from the previously constituted senate the role of providing advice and consent to presidential appointments.

The executive council shall have the power to nominate justices of the supreme court, which must be confirmed by the senate.
\end{quoting}

The crucial impact that Supreme court appointments have on constitutional law suggests that it ought to be debated by an adversarial body like the senate. However, the council seems to be a more fitting institution for the nominations than the president. For other appointments, the senate already gives the president a lot of leeway. The executive council would be better positioned to provide the “advice” part of this task, with their long terms and national constituency.

\begin{quoting}
A council vote attains a supermajority if at least three fifths of the members vote in the affirmative, or equivalently 5 members from a full council of 7. With a supermajority, the council has the following powers:
\end{quoting}

The following are very serious actions, and therefore we require the legitimacy of a supermajority. This provides a more moderate approach to the line item veto and a way to fight against unconstitutional legislation.

\begin{quoting}
With a supermajority the council may reduce the amount of any appropriation made by law, so long as the disbursements from that appropriation are under their power of review by law.
\end{quoting}

The second part here looks complex, but we don’t want them to be able to block interest payments and other government promises.

\begin{quoting}
With a supermajority, the council may declare any legislation unconstitutional or a threat to the constitutional system of government, and shall include their reasons for any such declaration. The legislation is immediately null and void. However, any member of congress has standing to challenge this reasoning at the supreme court which has the authority to evaluate and overturn the decision.
\end{quoting}

Given the council’s structure, they are in a better position than congress or even the president to check the increasing power of the unelected administrative state. The president can currently remove some appointments but would have a harder time changing policy directly\cite{Fairlie}. Even if that ability was given directly to the president, it would be difficult for a single person to wield that power in a thoughtful way that confers legitimacy. Therefore, this gives the president the ability to overturn administrative policy with the consent of the council, and if the council is nearly unanimous in extraordinary situations they can do it themselves.

\begin{quoting}
With either a supermajority or a simple majority that includes the president, the council may review any administrative policy originating in the executive branch, and to nullify them within 7 years. They may also remove administrative officers and employees who make policy through the same process. This clause does not remove any powers held by the president to act unilaterally in cases of policy and removal.
\end{quoting}

The rule from a supermajority without the president is designed to give the council a moderating pushback on presidential actions, which is necessary to check their new powers over the administrative state. The bar is set high to prevent the feeling of a continual coup over the president from the executive council.

The 7 year rule gives a chance for a full turnover of the executive council. Once that time period has passed, the rule is now precedent and must be changed through the usual channels, whether by law or by the procedures of the administrative state itself.

A bit of work is still needed to determine what counts as an administrative policy or a policy-making employee. More input from experts is needed in order to ensure that the application of this clause has the desired coverage.

\begin{quoting}
Executive councilors are subject to the federal impeachment process.

Executive councilors shall be well compensated as set out by law.
\end{quoting}

Why should the federal government provide these councilors with a high-level executive compensation package? This will ensure a strong pool of talent from many sectors of public and private life for any councilor election; and this strong talent pool will lead to better fiscal management.

Do not expect a heavily ideological group of councilors. If the system works well, we might attain talent from business, foreign affairs, politics, and other professions. We would also like these people to have an increased respect for an appropriate balance of power and to be long-term strategic thinkers.

\section{Electing the President}

\begin{quoting}
\textbf{Section 4}

A person who has been a citizen of the United States for 7 years, is not ineligible to that Office by reason of not being a native-born citizen of the United States.
\end{quoting}

We start here by removing the anachronistic “natural born citizen” clause which the founders included to prevent European royalty from coming over and capturing the government. The phrasing comes from Orin Hatch\cite{Somin_Hatch}, with a reduction of the 20 year requirement as convincingly argued by Ilya Somin\cite{Somin_NBC}.

\begin{quoting}
The number of electors for president shall be equal to the number of representatives to which the state and territory may be entitled, and shall be elected in the same manner as its representatives.
\end{quoting}

The electoral college is currently proportioned by adding the number of senate and house seats from each state. That formula depends entirely on the size of the house with respect to the senate which was selected for entirely different reasons than to affect presidential elector apportionment.

We now have a stronger executive council which uses state apportionment. Therefore we can drop the senate-based apportionment term from the presidential election, and make it based purely on population.

\begin{quoting}
If the electoral college fails to yield a majority vote for any candidate and a contingency election occurs in the house of representatives, that contingency election shall apportion one vote per representative.
\end{quoting}

The pattern is now complete. The president will be chosen on the same basis as the house of representatives, and the executive council will be chosen on the same basis as the senate. Both DC and the US territories will have representation in the electoral college. This supersedes the 23rd amendment.

This system removes one criticism of the electoral college, that regions are not represented uniformly based on their population. However, it does not create a national popular vote. Organizing voters into districts and constituencies is still valuable in its fault tolerance. In other words, it limits the effect of local anomalies.

There is a connection here with the recent interest in neural network architecture for intelligent software. These networks are organized into layers which are each filtered through an activation function which truncates one or both sides of its input. Because such networks are often found in nature, we can consider our electoral college not a shameful relic of the past, but as the neural network of democracy.

Of course, these ideas will be debated and revisited many times in the future. This is why this proposal encourages a vigorous debate over election law in the courtrooms and the marketplace of ideas.

\section{Summary}

This proposal is meant to raise several points about the design of governing institutions. The first is that finding a clear purpose for such institutions is of utmost importance about determining its structures and its powers. If such structures appear to be arbitrary, it will be of use to step back to the purpose and ask whether the structure helps it serve that perported purpose.

\begin{table}[ht]
\label{table:institutions}
\centering
\renewcommand{\arraystretch}{1.5}
\begin{tabular}{|c|c|c|c|c|}
\hline
\textbf{Institution} & \makecell{House of \\ Representatives} & Senate & Presidency & \makecell{Executive \\ Council}  \\
\hline
\textbf{Purpose} & Voice of the People & Voice of the States & Leadership & \makecell{Professionalism \\ Continuity} \\
\hline
\textbf{Branch} & Legislative & Legislative & Executive & Executive \\
\hline
\textbf{Term Length} & 2 Years & State Law & 4 Years & 6 Years \\
\hline
\textbf{Process} & Adversarial & Adversarial & Unitary & Cooperative \\
\hline
\textbf{Apportionment} & Population & States & Population & States  \\
\hline
\textbf{Electors} & Direct per District & State Law  & Electoral College & Electoral College  \\
\hline
\textbf{Winning Condition} & Plurality & State Law & Majority & Condorcet  \\
\hline
% you can continue adding rows in the same format
\end{tabular}
\caption{A summary of the proposed legislative and executive institutions and their properties. }
\end{table}

In table~\ref{table:institutions}, we summarize the four institutions laid out in this document and their various properties. In addition, each of these sintuttions is given unique powers. We've discussed some here, but i's open to argument [fix].

This work can be continued in several ways... [ask if it achieves balance]

[Final Paragraph ]

\section{Acknowledgements}

Aaron Bell, Jordan Marks

Cutrone (re: bicameralism)

ScheduleF
Blockwood (reforming the civil service)
Crowell (Administrative State)
Cooper
Amar

\begin{thebibliography}{20}
\bibitem{Tucker}Tucker, Jeffrey. \href{https://www.theepochtimes.com/repeal-the-17th-amendment-now\_4909126.html}{“Repeal the 17th Amendment Now.”} The Epoch Times, December 7, 2022.

\bibitem{Virginia}McInerney, Virginia. \href{https://lonang.com/commentaries/conlaw/federalism/repeal-seventeenth-amendment/}{“Why the 17th Amendment Needs to Be Repealed.”} LONANG Institute, March 27, 2020.

\bibitem{Hahn-Burkett}Hahn-Burkett, Tracy. \href{https://www.concordmonitor.com/What-is-the-Executive-Council-34817477}{“3-Minute Civics: What Is the Executive Council?”} Concord Monitor, June 23, 2020.

\bibitem{Khanna}Khanna, Parag. \href{https://qz.com/876260/seven-presidents-are-better-than-one-why-the-oval-office-needs-a-round-table}{“Seven Presidents Are Better than One: Why the Oval Office Needs a Round Table.”} Quartz, January 10, 2017.

\bibitem{Senate}\href{https://www.senate.gov/about/origins-foundations/senate-and-constitution.htm}{“About the Senate and the Constitution.”} U.S. Senate: About the Senate and the Constitution, September 1, 2022.

\bibitem{Eisinger}Eisinger, Vince. "Auxiliary Protections: Why the Founders' Bicameral Congress Depended on Senators Elected by State Legislatures." Touro L. Rev. 31 (2014): 231.

\bibitem{Federalist10}Madison, James. "The Federalist Papers, No. 10." November 22, 1787 (1787).
\bibitem{Federalist62}Madison, James. "The Federalist Papers, No. 62." Independent Journal, February 27 (1788).

\bibitem{Cutrone}Cutrone, Michael, and Nolan McCarty. "Does bicameralism matter?." (2008).

\bibitem{ScheduleF}Executive Order No. 13957. (2020) \href{https://www.
federalregister.gov/documents/2020/10/26/2020-23780/creating-schedule-f-in-the-excepted-service}{Creating schedule F in the excepted service}, October 21, 2020.

\bibitem{Timmins}Timmins, Annmarie. \href{https://www.nhpr.org/nh-news/2021-10-25/executive-council}{“The Behind-the-Scenes Power of N.H.’s Executive Council Is Now at Center Stage.”} New Hampshire Public Radio, October 25, 2021. 

\bibitem{Blockwood}Blockwood, James-Christian. \href{https://www.govexec.com/management/2023/04/lets-rethink-management-our-civil-service/385654/}{“Let’s Rethink the Management of Our Civil Service.”} Government Executive, April 28, 2023.

\bibitem{New Hampshire Constitution}Constitution of New Hampshire, Part 2, Article 56

\bibitem{Kaufman}Kaufmann, Bruno. “\href{https://www.swissinfo.ch/eng/business/the-strengths-of-a--weak--swiss-government/48483858. }{The Strengths of a ‘weak’ Swiss Government}.” SWI swissinfo.ch, May 5, 2023. 

\bibitem{30000}“\href{https://thirty-thousand.org/the-house-of-representatives-is-scalable/}{Learn Why a Larger House Will Be Much More Productive and Enable the Representatives to Better Serve Their Constituents}.” thirty-thousand.org, June 6, 2022.

\bibitem{Allen}Allen, Danielle. “Opinion | \href{https://www.washingtonpost.com/opinions/2023/03/28/danielle-allen-democracy-reform-house-representatives-districts/}{Just How Big Should the House Be? Let’s Do the Math}.” The Washington Post, May 23, 2023. 

\bibitem{Schulze}Schulze, Markus. "The Schulze method of voting." arXiv preprint arXiv:1804.02973 (2018).

\bibitem{Purnick}Purnick, Joyce. “\href{https://www.nytimes.com/1986/02/16/weekinreview/the-board-of-estimate-and-its-critics.html}{The Board of Estimate and Its Critics}.” The New York Times, February 16, 1986.

\bibitem{Crowell}"\href{https://www.crowell.com/a/web/bomv5ATK9LZPNrBA51skWq/4TtiyY/Regulatory-Forecast-2020-Administrative-Law-Crowell-Moring.pdf. }{Administrative Law – The Supreme Court and the President Rein in the `Administrative State`}." Regulatory Forecast 2020, 2020. 

\bibitem{Somin_Hatch}Somin, Ilya. \href{https://reason.com/volokh/2020/08/16/orrin-hatchs-constitutional-amendment-to-abolish-the-natural-born-citizen-clause/}{“Orrin Hatch’s Constitutional Amendment to Abolish the Natural Born Citizen Clause.”} Reason.com, August 29, 2020.

\bibitem{Somin_NBC}Somin, Ilya. \href{https://reason.com/volokh/2020/08/14/why-we-should-abolish-the-requirement-that-the-president-must-be-a-natural-born-citizen/}{“Why We Should Abolish the Requirement That the President Must Be a ‘Natural Born Citizen’ [Updated].”} Reason.com, August 29, 2020. 

\bibitem{Fairlie}Fairlie, John A. "The Administrative Powers of the President." Michigan Law Review (1903): 190-210.

\bibitem{Cooper}Cooper, Charles J. \href{https://www.nationalaffairs.com/publications/detail/confronting-the-administrative-state}{“Confronting the Administrative State.”} National Affairs, Fall 2015.

\bibitem{Brewer}Brewer, Simon. \href{https://law.yale.edu/mfia/case-disclosed/back-basics-why-partisan-gerrymandering-violates-first-amendment. }{“Back to Basics: Why Partisan Gerrymandering Violates the First Amendment.”} Yale Law School, Media Freedom \& Information Access Clinic, March 12, 2019.

\bibitem{Amar}Amar, Akhil Reed. "Philadelphia Revisited: Amending the Constitution Outside Article V." U. Chi. L. Rev. 55 (1988): 1043.

\bibitem{Madison}Madison, James. Original Notes on Debates in the Congress of Confederation, 1787.

\bibitem{Board_of_Estimate}Board of Estimate of NYC v. Morris, 489 U.S. 688 (1989)

\bibitem{Ruger}Ruger, William, and Jason Sorens. Freedom in the 50 states, 2013 edition: An index of personal and economic freedom. The Mercatus Center at George Mason University, 2013.

\bibitem{Schiller}Schiller, Wendy J., and Charles Stewart III. \href{https://www.brookings.edu/wp-content/uploads/2016/06/Schiller_17th-Amendment_v7.pdf}{"The 100th Anniversary of the 17th Amendment: A Promise Unfulfilled?"}. Issues in Governance Studies at the Brookings Institute. May, 2013

\bibitem{Smith}Smith, Terry. \href{https://www.latimes.com/archives/la-xpm-2010-oct-22-la-oew-smith-17th-amendment-20101022-story.html}{“Why We Have, and Should Keep, the 17th Amendment.”} Los Angeles Times, October 22, 2010.

\bibitem{Lublin}Lublin, Joann S. “Smaller Boards Get Bigger Returns.” The Wall Street Journal, August 29, 2014. https://www.wsj.com/articles/smaller-boards-get-bigger-returns-1409078628. 

\bibitem{Mason}Rutland, Robert. "The Papers of George Mason", 3 vols. The University of North Carolina Press. Chapel Hill, NC. 1970.

\bibitem{Mason_Objection}Mason, George. "Objections to the Constitution of Government Formed by the Convention." Allen and Lloyd, The Essential Antifederalist (1787): 12.

\end{thebibliography}

This document along with revisions is posted at github as https://github.com/maxsklar/great-compromise-ammendment. See readme for contact information. No funds, grants, or other support was received.
\end{document}
